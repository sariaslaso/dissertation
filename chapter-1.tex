\chapter{Introduction}
\label{cha:introduction}

% ADD REFS WHERE IT IS INDICATED
% OUTLINE ATI PARAGRAPHS 

Atomic and molecular systems exposed to strong external fields are
extensively explored. The H$_{2}$O molecule, which is of interest in
this work, has served as reference of nonlinear molecule under strong
external fields and has received considerable attention in
ion-molecule collision
studies~(see~\cite{horbatsch_2012col,illescas_2013} and references
therein) due to its relevance in applied fields, such as radiation
therapy.  The multicentre nature of its potential makes the water
molecule an attractive and challenging problem, and diverse
approximations have been implemented in order to learn about its
molecular structure as well as about the impact of external
perturbations such as strong dc fields and high-intensity laser
fields~\cite{Jagau_manybody_H2O,Toru_weakF_H2O,Krause2015_CAP_H2O,Zhao_2011_H2Otunneling,Petretti_H2O_laser}.

High-intensity laser-atom interactions are the origin of phenomena
such as above threshold ionization~(\textsc{ati}), which reveals an
atom may absorb many more photons than the minimum necessary for
ionization~\cite{ATI1979}.
% rephrase this sentence (ATI chaper by Becker)
Under the effect of an intense laser field, an atom that is initially
in its ground state gets ionized at some given time followed by the
ejection of an electron that interacts with the laser field once it is
promoted to the continuum and eventually rescatters to within the
vicinity of the binding potential as the external field changes
direction. As a consequence of this interaction, an \textsc{ati}
spectrum consisting of a sequence of peaks separated by the photon
energy is generated. The study of this spectrum has been of increased
interest as it reveals features that describe the mechanism of
interaction of an atom with an external
field~\cite{BeckerRescattering_2018,Becker_ati2002}.

Events of strong dc field ionization of the water molecule valence
orbitals and laser-induced ionization of atoms constitute the essence
of this work. This study involves deriving an effective potential that
reproduces the symmetry properties of each molecular orbital as an
initial step in the calculation of the Stark resonance parameters for
the water molecule under an external static electric field.
% rephrase this mentioning that solving a tdse is computationally
% taxing and an intricate problem
The interaction of a strong laser field with atoms, in particular, the
phenomenon of \textsc{ati} is addressed within a framework that allows
to uncover the underlying physics without having to resort to
computationally demanding tools such as solving the time-dependent
Schr\"{o}dinger equation. A generalization of the strong-field
approximation~(\textsc{sfa})~\cite{KeldyshSFA} is implemented within
two independent frameworks: a semi-classical approach involving
quantum paths~\cite{KopoldOptComm2000} and a numerical evaluation of
the exact \textsc{sfa} results~\cite{Kopold_1997sfa}.

% develop the idea that was followed on each problem to obtain our
% results

% dc ionization of H2O
% other approaches, more complex or intricate, computationally taxing, etc
% Effective potential
% ECS
% numerical calculation of resonance parameters

Since the work of Ellison and Shull~\cite{EllisonShullh2o_1955},
numerous theoretical studies that attempt to formulate an accurate
description of the H$_{2}$O molecule have been reported. The
self-consistent field~(\textsc{scf}) method introduced by
Roothaan~\cite{Roothaan_1951}, which allows to represent the
Hartree-Fock orbitals of a molecule as expansions of basis functions,
has been widely implemented in the study of the ground state and
symmetry properties of H$_{2}$O. Within the \textsc{scf} framework,
the molecular orbitals wave functions have been approximated using
multicentre Slater-type atomic orbitals as basis
sets~\cite{Reeves_nature_1956,natureH2O_1960,Pitzer_1968,Pitzer_1970},
as well as Gaussian basis
functions~\cite{gaussianH2O_1965,Neumann_gaussian_1968}. On a similar
note, one-centre expansions were resorted to as attempts to bypass the
difficulty of evaluating multicentre
integrals~\cite{Moccia_1964,oneCentre_1961,Parr_JCP_1960}, however,
inherent to this method is the additional difficulty of needing a more
extensive set of basis functions. As an example, Moccia introduced an
expansion of the H$_{2}$O \textsc{mo}s in terms of Slater-type
functions all centred at the nucleus of the oxygen
atom~\cite{Moccia_1964}. This work, in which the author determined the
expansion coefficients of a linear combination consisting of $28$
Slater-like functions by means of Roothaan's \textsc{scf} method,
obtained wave functions that led to remarkably accurate values for the
total energy and the ground state configuration of H$_{2}$O.

% 1.mention that Moccia results is the starting point of the current work
% 2.solve system of PDE
% 3.Implementing a modified exterior complex scaling...

Solving the time-dependent Schr\"{o}dinger equation~(\textsc{tdse}) in
the study of static-field ionization rates could perhaps seem like a
logical effort that leads to highly accurate solutions. This approach
has been implemented, within the framework of Hermitian quantum
mechanics, in tunneling calculations for the helium
atom~\cite{static_tdse_He,static_tdse_He_method} in which the
two-electron Schr\"{o}dinger equation was solved and the results were
in good agreement with previous
calculations~\cite{static_He_scrinzi}. However, obtaining a numerical
solution to the many-body \textsc{tdse}, corresponding to molecular
Stark resonances, remains a challenging problem even for a small
number of particles.

% non-Hermitian molecular static-field ionization rates
% coupled cluster
% CAP
% ECS
% refs from Patrick's MSc thesis (about ECS)
% modified ECS (our approach)

Time-independent approaches, which imply a non-Hermitian
representation of quantum mechanics by means of a complex-variable
representation of the molecular Hamiltonian, have established useful
alternatives in the study of molecular static-field ionization. Within
a complex-variable framework, the Stark resonance parameters induced
by an external field are associated with a discrete set of complex
eigenvalues. Among the proposed methods, coupled-cluster~(\textsc{cc})
calculations of molecular strong-field ionization provided accurate
results for the Stark and static-field ionization rates of several
molecules~\cite{Jagau_manybody_H2O}. In this work, the author combined
\textsc{cc} methods~\cite{cc_method} with complex basis functions,
consisting of basis sets of atom-centred Gaussian functions with a
complex-scaled exponent, and computed molecular Stark resonances
linked to complex eigenstates of H$_{2}$O for different orientations
of the external dc field.

% CC advantage
%A particular advantage of a CC treatment of molecular Stark resonances
%is that all ionization channels can be computed as eigenstates of the
%same Hamiltonian in a biorthogonal representation through the
%equation-of-motion (EOM) CC formalism.48,49 Thus, their
%characterization through Dyson orbitals is straightforward.50,51 Also,
%the CC formalism for molecular properties can be applied to compute
%moments of the electronic charge distribution, which provides further
%insight into the ionization process

Among alternative well-established methods to compute resonance states
are those of complex scaling~(\textsc{cs})~\cite{complexScalingSimon}
and complex absorbing potentials~(\textsc{cap}s)~\cite{RissMeyer_1993}
in which the Hamiltonian is extended analitically into the complex
plane by an artificial local potential formulated to absorb the
diverging tale of the resonance wave functions at the boundaries of a
finite volume. The resonance parameters are then obtained from the
square-integrable eigenfunctions of the modified non-Hermitian
Hamiltonian without modifying the boundary conditions of the
problem. Extensions of these methods, such as the exterior complex
scaling~(\textsc{ecs})~\cite{Simon_1979}, have been introduced in
studies of ionization of molecular
structures~\cite{ScrinziJChemPhys_ECS,ScrinziJPhysB_ECS} and to
determine numerical solutions of the \textsc{tdse} for
strong-field-induced dynamics in atoms and
molecules~\cite{Krause_2014,ecsScrinzi,ecsRuiz}.

% Scrinzi, PRA81(2010)
% None of these methods appeared to be completely satisfactory:
% absorbing masks and the closely related CAP’s require careful
% adaptation to a given problem and a comparatively large absorption
% range, but still do not allow perfect absorption.  As to ECS, the
% two recent numerical studies have cast doubt on the efficiency [5]
% and maybe even the fundamental correctness the method in numerical
% practice [4]: rather poor accuracies, problems with numerical
% stability, apparent fundamental limitations of long-range
% absorption, and also poor efficiency were reported.

A modified exterior complex scaling approach in which the radial
coordinates are gradually continued into the complex plane, the method
implemented in the current work, is introduced with the aim of
studying the field-ionization properties of the H$_{2}$O valence
orbitals. This method permitted to formulate the problem of H$_{2}$O
static-field ionization as a system of partial differential
equations~(\textsc{pde}) in which the Stark resonance parameters were
obtained via the complex eigenenergies of the \textsc{pde} system.



 %Tunneling data on static-field
%ionization of H$_{2}$O can be found in calculations on its tunneling
%ionization regime within the weak-field asymptotic
%theory~\cite{Toru_weakF_H2O,Toru_weakF_molec}. In this work the
%authors presented an accurate description of the asymptotic behaviour
%of the H$_{2}$O wave functions within the single-active-electron
%approximation.






% laser-atom interaction via SFA and SPA
% mention previous complete approaches

% solving Schr.eq. intricate problem etc etc
% strong field approximation
% saddle point approximation

% see Becker_2018 review article

% outline the contents of the rest of the chapters
This dissertation...



Atomic units $(\hbar = m = -e = 4\pi\epsilon_{0} = 1)$ are used
throughout.
































%%% Local Variables:
%%% mode: latex
%%% TeX-master: "thesis"
%%% End:
