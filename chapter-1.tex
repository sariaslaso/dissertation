\chapter{Introduction}
\label{cha:introduction}

% ADD REFS WHERE IT IS INDICATED
% OUTLINE ATI PARAGRAPHS 


Atomic and molecular systems exposed to strong external fields have
been explored extensively. The H$_{2}$O molecule, which is of interest
in this work, has served as a reference for nonlinear molecules under
strong external fields and has received considerable attention in
ion-molecule collision
studies~(see~\cite{horbatsch_2012col,illescas_2013} and references
therein) due to its relevance in applied fields, such as radiation
therapy.  The multicentre nature of its potential makes the water
molecule an attractive and challenging problem, and diverse
approximations have been implemented in order to learn about its
molecular structure as well as its interaction with external
perturbations such as strong dc fields and high-intensity laser
fields~\cite{Jagau_manybody_H2O,Toru_weakF_H2O,Krause2015_CAP_H2O,Zhao_2011_H2Otunneling,Petretti_H2O_laser}.

High-intensity laser-atom interactions are the origin of phenomena
such as \textsc{ati}, which reveals that an atom may absorb many more
photons than the minimum necessary for ionization~\cite{ATI1979}.
% rephrase this sentence (ATI chaper by Becker)
Under the effect of an intense laser field, an atom that is initially
in its ground state gets ionized at some given time followed by the
ejection of an electron that interacts with the laser field once it is
promoted to the continuum and eventually rescatters to within the
vicinity of the binding potential as the external field changes
direction. As a consequence of this interaction, an \textsc{ati}
spectrum consisting of a sequence of peaks separated by the photon
energy is generated. The study of this spectrum has been of increased
interest as it reveals features that describe the mechanism of
interaction of an atom with an external
field~\cite{BeckerRescattering_2018,Becker_ati2002}.

Events of strong dc field ionization of the water molecule valence
orbitals and laser-induced ionization of atoms constitute the essence
of this work. This study involves deriving an effective potential that
reproduces the symmetry properties of each \textsc{mo} as an initial
step in the calculation of the Stark resonance parameters for the
water molecule under an external static electric field.
% rephrase this mentioning that solving a tdse is computationally
% taxing and an intricate problem
The interaction of a strong laser field with atoms, in particular, the
phenomenon of \textsc{ati} is addressed within a framework that allows
to uncover the underlying physics without having to resort to
computationally demanding tools such as solving the time-dependent
Schr\"{o}dinger equation~(\textsc{tdse}). A generalization of the
\textsc{sfa}~\cite{KeldyshSFA} is implemented within two independent
frameworks for the case of a zero-range potential: a semi-classical
approach involving quantum paths~\cite{KopoldOptComm2000} and a
numerical evaluation of the exact \textsc{sfa}
results~\cite{Kopold_1997sfa}.

% develop the idea that was followed on each problem to obtain our
% results

% dc ionization of H2O
% other approaches, more complex or intricate, computationally taxing, etc
% Effective potential
% ECS
% numerical calculation of resonance parameters

Since the work of Ellison and Shull~\cite{EllisonShullh2o_1955},
numerous theoretical studies that attempt to formulate an accurate
description of the H$_{2}$O molecule have been reported. The
self-consistent field~(\textsc{scf}) method introduced by
Roothaan~\cite{Roothaan_1951}, which allows to represent the
Hartree-Fock~(\textsc{hf}) orbitals of a molecule as expansions of
basis functions, has been widely implemented in the study of the
ground state and symmetry properties of H$_{2}$O. Within the
\textsc{scf} framework, the \textsc{mo}s wave functions have
been approximated using multicentre Slater-type atomic orbitals as
basis
sets~\cite{Reeves_nature_1956,natureH2O_1960,Pitzer_1968,Pitzer_1970},
as well as Gaussian basis
functions~\cite{gaussianH2O_1965,Neumann_gaussian_1968}.

Additionally, one-centre expansions were applied in order to bypass
the difficulty of evaluating multicentre
integrals~\cite{Moccia_1964,oneCentre_1961,Parr_JCP_1960}. However,
inherent to this method is the additional difficulty of needing a more
extensive set of basis functions. As an example, Moccia introduced an
expansion of the H$_{2}$O \textsc{mo}s in terms of Slater-type
functions all centred at the nucleus of the oxygen
atom~\cite{Moccia_1964}. This work, in which the author determined the
expansion coefficients of a linear combination consisting of $28$
Slater-like functions by means of Roothaan's \textsc{scf} method,
obtained wave functions that led to remarkably accurate values for the
total energy of the ground state configuration of H$_{2}$O.

% 1.mention that Moccia results is the starting point of the current work
% 2.solve system of PDE
% 3.Implementing a modified exterior complex scaling...

Solving the \textsc{tdse} in the study of static-field ionization
rates could perhaps seem like a logical effort that leads to highly
accurate solutions. This approach has been implemented, within the
framework of Hermitian quantum mechanics, in tunneling calculations
for the helium atom~\cite{static_tdse_He,static_tdse_He_method} in
which the two-electron Schr\"{o}dinger equation was solved and the
results were in good agreement with previous
calculations~\cite{static_He_scrinzi}. However, obtaining a numerical
solution to the many-body \textsc{tdse}, corresponding to molecular
Stark resonances, remains a challenging problem even for a small
number of particles.

% non-Hermitian molecular static-field ionization rates
% coupled cluster
% CAP
% ECS
% refs from Patrick's MSc thesis (about ECS)
% modified ECS (our approach)

Time-independent approaches, which make use of a non-Hermitian
formulation of quantum mechanics by means of a complex-variable
representation of the molecular Hamiltonian, have established useful
alternatives in the study of molecular static-field ionization. Within
a complex-variable framework, the Stark resonance parameters induced
by an external field are associated with a discrete set of complex
eigenvalues. Among the proposed methods, coupled-cluster~(\textsc{cc})
calculations of molecular strong-field ionization provided accurate
results for the Stark and static-field ionization rates of several
molecules~\cite{Jagau_manybody_H2O}. In this work, the author combined
\textsc{cc} methods~\cite{cc_method} with complex basis functions,
consisting of basis sets of atom-centred Gaussian functions with a
complex-scaled exponent, and computed molecular Stark resonances
linked to complex eigenstates of H$_{2}$O for different orientations
of the external dc field.

% CC advantage
%A particular advantage of a CC treatment of molecular Stark resonances
%is that all ionization channels can be computed as eigenstates of the
%same Hamiltonian in a biorthogonal representation through the
%equation-of-motion (EOM) CC formalism.48,49 Thus, their
%characterization through Dyson orbitals is straightforward.50,51 Also,
%the CC formalism for molecular properties can be applied to compute
%moments of the electronic charge distribution, which provides further
%insight into the ionization process

Among alternative well-established methods to compute resonance states
are those of complex scaling~(\textsc{cs})~\cite{complexScalingSimon}
and complex absorbing potentials~(\textsc{cap}s)~\cite{RissMeyer_1993}
in which the Hamiltonian is extended analytically into the complex
plane by an artificial local potential formulated to absorb the
diverging tail of the resonance wave functions at the boundaries of a
finite volume. The resonance parameters are then obtained from the
square-integrable eigenfunctions of the modified non-Hermitian
Hamiltonian with absorbing boundary conditions. Extensions of these
methods, such as the exterior complex
scaling~(\textsc{ecs})~\cite{Simon_1979}, have been introduced in
studies of ionization of molecular
structures~\cite{ScrinziJChemPhys_ECS,ScrinziJPhysB_ECS}, and to
determine numerical solutions of the \textsc{tdse} for
strong-field-induced dynamics in atoms and
molecules~\cite{Krause_2014,ecsScrinzi,ecsRuiz}.

% Scrinzi, PRA81(2010)
% None of these methods appeared to be completely satisfactory:
% absorbing masks and the closely related CAP’s require careful
% adaptation to a given problem and a comparatively large absorption
% range, but still do not allow perfect absorption.  As to ECS, the
% two recent numerical studies have cast doubt on the efficiency [5]
% and maybe even the fundamental correctness the method in numerical
% practice [4]: rather poor accuracies, problems with numerical
% stability, apparent fundamental limitations of long-range
% absorption, and also poor efficiency were reported.

A modified exterior complex scaling approach in which the radial
coordinates are gradually continued into the complex plane, the method
implemented in the current work, is introduced with the aim of
studying the field-ionization properties of the H$_{2}$O valence
orbitals. This method permitted to formulate the problem of H$_{2}$O
static-field ionization as a system of partial differential
equations~(\textsc{pde}) in which the Stark resonance parameters were
obtained via the complex eigenenergies of the \textsc{pde}
system~\cite{sarias_2016,sarias_2017}.

%Tunneling data on static-field ionization of H$_{2}$O can be found in
%calculations on its tunneling ionization regime within the weak-field
%asymptotic theory~\cite{Toru_weakF_H2O,Toru_weakF_molec}. In this
%work the authors presented an accurate description of the asymptotic
%behaviour of the H$_{2}$O wave functions within the
%single-active-electron approximation.


% laser-atom interaction via SFA and SPA
% mention previous complete approaches

% solving Schr.eq. intricate problem etc etc
% strong field approximation
% saddle point approximation (quantum paths)

% see Becker_2018 review article

In what follows, the topic of strong-field laser-atom interactions is
addressed. Numerous formulations have been introduced which aim to
understand the physics behind the complicated structure of the
\textsc{ati} spectrum.  The Keldysh theory of strong-field
approximation is one of the pioneering works that properly accounts
for tunneling ionization~\cite{KeldyshSFA}, and produces accurate
electron spectra for \textsc{ati} for relatively low
energies. However, the \textsc{sfa} in its early versions failed to
provide a comprehensive description of the ionization spectrum, in
particular, to account for the extended plateau at higher energies
that was first observed in~\cite{Paulus_1994plateau} for the
\textsc{ati} spectrum of rare gases in strong laser pulses.

Extensions of the Keldysh theory have provided a deeper understanding
of laser-atom interactions and revealed the underlying mechanism that
gives origin to the \textsc{ati} plateau: rescattering. In order to
incorporate rescattering events, subsequent models needed to allow the
freed electron to interact once more with the ion. A semiclassical
three-step model which incorporates the effect of rescattering, in
which the atomic potential is considered a perturbation while the
electron propagates in the laser field,
~\cite{Becker_rescattering1994,Becker_1994plateau_classical} was
successful in revealing the complicated angular distributions of the
\textsc{ati} spectrum. Additional attempts that incorporate
rescattering in the context of the Keldysh approach were implemented
by means of Coulomb-Volkov solutions~\cite{Kaminski_1997}.

%paragraph about SPA and quantum paths

An interesting interpretation of the laser-atom interactions is the
one offered by the path-integral formulation of quantum
mechanics~\cite{RevModPhysFeynman}. The concept of a quantum path that
connects the initial and final state of a system, and combines the
tunneling of an electron with its subsequent semiclassical propagation
in the laser field, has been widely applied at explaining subtle
features of \textsc{ati}~\cite{LewScience2001}. In addition,
incorporating the path-integral principles within the framework of
\textsc{sfa} calculations, has allowed to successfully reproduce the
structure of \textsc{ati} spectra that can be connected with
interferences of the contributions of a finite number of quantum
trajectories~\cite{KopoldOptComm2000}.


% outline the contents of the rest of the chapters

This dissertation addresses two separate topics, the static-field
ionization of the H$_{2}$O molecule and the phenomenon of \textsc{ati}
within the framework of an extended
\textsc{sfa}. Chapter~\ref{cha:scf_h2o} presents the necessary
background of the \textsc{hf} formulation that sets the basis for the
\textsc{scf} calculation of the H$_{2}$O orbital
energies~\cite{Moccia_1964} used as reference in the study of the
water molecule under the effect of an external
field. Chapter~\ref{ch:dc_h2o} contains two independent studies of the
dc field ionization of the H$_{2}$O valence orbitals, $1b_{1}$,
$1b_{2}$, and $3a_{1}$ within the framework of non-Hermitian quantum
mechanics. A modified exterior complex scaling and a complex-absorbing
potential combined with a partial-wave expansion of a previous model
potential are implemented. The topic of laser-atom interactions is
approached in Chapters~\ref{cha:ati}
and~\ref{cha:sp_approx}. Chapter~\ref{cha:ati} contains a study of
convergence of the \textsc{ati} spectrum associated with an improved
Keldysh model~\cite{Kopold_1997sfa} within the limits of a zero-range
potential in which events of direct ionization and ionization with
rescattering are treated separately. A saddle-point approximation is
implemented in Chapter~\ref{cha:sp_approx} to the problem of
evaluating the \textsc{ati} spectrum of a model-helium atom in a
strong laser field in terms of quantum orbits. Finally, conclusions of
this work are presented in Chapter~\ref{ch:conclusions}.

Atomic units $(\hbar = m = -e = 4\pi\epsilon_{0} = 1)$ are used
throughout unless otherwise indicated.
































%%% Local Variables:
%%% mode: latex
%%% TeX-master: "thesis"
%%% End:
