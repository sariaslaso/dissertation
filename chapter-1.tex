\chapter{Introduction}
\label{cha:introduction}


Atomic and molecular systems exposed to strong external fields are
extensively explored.
% rephrase this sentence (paper introduction)
The water molecule, in particular, has received considerable attention
in ion-molecule collision studies~[\textcolor{red}{REFS}] due to its
relevance in applied fields, such as radiation therapy. The
multicentre nature of its potential makes the water molecule an
attractive and challenging problem, and a diversity of approximations
have been implemented in order to learn about its molecular structure
as well as about the impact of external perturbations such as
ion-molecule-collisions~[\textcolor{red}{REFS}], strong dc
fields~[\textcolor{red}{REFS}] and high-intensity laser
fields~[\textcolor{red}{REFS}].

High-intensity laser-atom interactions are the origin of phenomena
such as above threshold ionization~(\textsc{ati}), in which reveals an
atom may absorb many more photons than the minimum necessary for
ionization~[\textcolor{red}{REFS}].
% rephrase this sentence (ATI chaper by Becker)
Under the effect of an intense laser field, an atom that is initially
in its ground state gets ionized at some given time followed by the
ejection of an electron that interacts with the laser field once it is
promoted to the continuum and eventually rescatters to within the
vicinity of the binding potential as the external field changes
direction. As a consequence of this interaction, an \textsc{ati}
spectrum consisting of a sequence of peaks separated by the photon
energy is generated. The study of this spectrum has been of increased
interest as it reveals features that describe the mechanism of
interaction of an atom with an external field~[\textcolor{red}{REFS}].

Events of strong dc field ionization of the water molecule valence
orbitals and laser-induced ionization of atoms describe the essence of
this work. This study involves deriving an effective potential that
reproduces the symmetry properties of each molecular orbital as an
initial step in the calculation of the Stark resonance parameters for
the water molecule under an external static electric field.
% rephrase this mentioning that solving a tdse is computationally
% taxing and an intricate problem
The interaction of a strong laser field with atoms, in particular, the
phenomenon of \textsc{ati} is addressed within a framework that allows
to uncover the underlying physics without having to resort to
computationally demanding tools such as solving the time-dependent
Schr\"{o}dinger equation. A generalization of the strong-field
approximation~(\textsc{sfa})~[\textcolor{red}{REFS}] is implemented
within two independent frameworks: a semi-classical approach involving
quantum paths and a numerical evaluation of the exact \textsc{sfa}
results.


% develop the idea that was followed on each problem to obtain our
% results



Atomic units $(\hbar = m = -e = 4\pi\epsilon_{0} = 1)$ are used
throughout.
































%%% Local Variables:
%%% mode: latex
%%% TeX-master: "thesis"
%%% End:
