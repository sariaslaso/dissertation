\chapter{Conclusions}
\label{ch:conclusions}

In this dissertation we have developed a study of the ionization of
the H$_{2}$O molecule valence orbitals under an electric dc field in
which \textsc{nhqm} techniques have been implemented. Additionally, a
systematic study of the \textsc{ati} spectrum for a model-He atom was
carried out following an extension of the \textsc{sfa}. In the
following subsections more detailed and summarized conclusions are
offered for these studies.

\section{H$_{2}$O in an external dc field}
\label{ch:h2o_results}

\subsection*{$\mathrm{PDE}$ system with a modified $\mathrm{ECS}$}

% describe the approach in a few sentences

% state that for the 1b1 and 1b2 mos, a spherical effective potential
% was used, and for the 3a1 mo a dependen with the polar angle was
% included 

We have carried out a study of the H$_{2}$O valence orbitals,
$1b_{1}$, $1b_{2}$, and $3a_{1}$, in the presence of an external
electric dc field applied along the symmetry axis of the molecule,
$\mathbf{F} = F_{0}\hat{z}$. As a starting point, an orbital-dependent
effective potential was extracted from a self-consistent solution for
the H$_{2}$O \textsc{mo}s expressed as an expansion on a basis of
single-centre Slater-type orbitals~\cite{Moccia_1964}. The $1b_{1}$
and $1b_{2}$ molecular orbitals were approximated by orbitals in a
spherically symmetric potential and expressed as linear combinations
of $2p_{x}$ and $2p_{y}$ orbitals, respectively. In the case of the
$3a_{1}$ orbital, $s-p$ type Slater orbitals were considered in
addition to retaining the $2p_{z}$ parts of the \textsc{mo} expansion.

A modified \textsc{ecs} technique was applied to the radial
coordinates in the Schr\"{o}dinger equation with orbital-dependent
effective potential for each \textsc{mo} as part of solving the
complex eigenvalue problem to obtain the Stark resonance
parameters. By means of a phase factor the scaling was turned on
gradually beyond some distance from the origin of coordinates. The
\textsc{ecs} parameters, $r_{\mathrm{s}}$ and $\Delta r$, as well as
the asymptotic scaling angle, $\chi_{\mathrm{s}}$, were optimized in
order to guarantee a minimum of two to three significant digits in the
solutions. The tunneling ionization and over-barrier ionization
regimes were explored by finding a numerical solution to the
\textsc{pde} system resulting from separating the complex-valued wave
function into real and imaginary parts~\cite{sarias_2016,sarias_2017}.


% 1b1/1b2 results
The resonance parameters that describe the ionization process,
resonance position and width, were explored over a wide range of
electric field strengths. It was demonstrated how an increase of the
field strength beyond a critical point in the over-barrier region
leads to a crossing between the ionization rates of the $1b_{1}$ and
$1b_{2}$ orbitals.
% THIS IS NOT SHOWN IN THE RESULTS SO MAYBE NOT MENTION IN CONCLUSIONS
%Additional observations of the spherical effective
%potential for different field strengths were carried out in order to
%shed some light on the interpretation of this behaviour.

% 3a1 results
For the $3a_{1}$ \textsc{mo}, the orientation of the external field
was chosen in two directions, that is the dc field pointing away or
towards the oxygen atom fixed at the origin. The relationship of the
resonance parameters (position and width) to the neighbouring valence
orbitals $1b_{1}$ and $1b_{2}$ was explored. Interestingly, the
$3a_{1}$ orbital was found to ionize more easily than $1b_{1}$ or
$1b_{2}$ irrespective of the field direction along $\hat{z}$.

These calculations should serve as motivation for further studies of
molecular orbitals of water using more sophisticated wave
functions. It would be interesting to extend this work to deal with
laser fields for practical applications. For the hydrogen molecular
ion this was done using Floquet theory~\cite{Tsog_H2mol_ac_2013},
where some parallels were found between monochromatic \textsc{ac} and
the \textsc{dc} cases. Experimental observations of strong-field
ionization of water vapour are available for short, intense laser
pulses~\cite{exp_h2o_laser_2008,exp_h2o_laser_2014}. One needs to
solve the time-dependent Schr\"{o}dinger equation for realistic
simulations of these~\cite{Farrell_2011,Falge_2010}.


\subsection*{Partial-wave approach with a $\mathrm{CAP}$}
% partial-wave + CAP approach

Ionization of H$_{2}$O valence orbitals exposed to an external dc
field has been addressed from an alternative perspective as well. The
hydrogen components of a three-centre model potential introduced in a
study of ion collisions with water
molecules~\cite{illescas_modelV_2011} were expressed as a partial-wave
expansion in a basis of spherical harmonics~\cite{marko_partialwave}
truncated at $l = l_{\mathrm{max}}$. This representation of the model
potential in terms of partial waves resulted in a system of radial
equations of dimensions proportional to the size of the expansion. The
system of coupled ordinary differential equations for the different
$(l,m)$ channels was solved numerically by implementing a
finite-element \emph{Mathematica} method. The precision of the
calculations was modified accordingly as the number $l_{\mathrm{max}}$
increased and, consequently, the number of $(l,m)$ channels.

The field-free problem was addressed and orbital energies for the
valence \textsc{mo}s were obtained for increasing sizes of the
partial-wave expansion. In a comparison with the solutions for the
model potential as quoted in a Gaussian orbital basis
approach~\cite{illescas_2015}, the partial-wave expansion results were
found to approximate the eigenvalues from above. The outermost orbital
$1b_{1}$, with its density perpendicular to the molecular plane and
little overlap with the hydrogen atoms, indicated a fast convergence
with $l_{\mathrm{max}}$. On the other hand, for the more deeply bound
\textsc{mo} $1b_{2}$, with a considerable amount of electron density
along each O$-$H bond, convergence with $l_{\mathrm{max}}$ was
noticeably slower. Comparison with the eigenvalues obtained for an
exchange-only \textsc{opm} calculation~\cite{opm_2007} was shown as
well.

The complex-valued resonance energies were computed by including a
quadratic \textsc{cap} that was turned on at a distance such that the
effective potential had reached its simple asymptotic form. The
eigenvalues were determined by implementing the Riss-Meyer correction
scheme~\cite{RissMeyer_1993}. Comparison of the calculated dc shifts
and decay rates with the previous results based on a single-centre
\textsc{scf} orbital dependent effective
potential~\cite{sarias_2016,sarias_2017}, for which azimuthal symmetry
was assumed, points to the fact that the partial-wave expansion of the
model potential has more geometric flexibility in the resonance
solution, and therefore shows more prominent features in the behaviour
of dc shift and decay rate $\Gamma$ as a function of the field
strength and field orientation.


\section{Above threshold ionization for laser-atom interactions}
\label{ch:ati_results}

We have investigated the phenomenon of \textsc{ati} for a model-He
atom in the presence of a strong linearly polarized laser field. A
generalization of the Keldysh formalism of
\textsc{sfa}~\cite{KeldyshSFA}, in which rescattering of an electron
with its parent ion is considered, has been explored in the limiting
case of a zero-range potential. A systematic study of the numerical
convergence of the \textsc{ati} spectrum has been carried out by
evaluating the compact expression for ionization of an atom into a
scattering state~\cite{Kopold_1997sfa} with asymptotic momentum
$\mathbf{p}$. Several parameters associated with the numerical
precision of the calculation, such as the working precision, were
tuned up in order to reduce the error when evaluating the relevant
quadrature.

In addition to the numerical evaluation of the compact expression for
the transition amplitude, an alternative analysis based on the concept
of quantum paths~\cite{KopoldOptComm2000} was implemented. This
approach permitted to generate the \textsc{ati} spectrum by means of a
saddle-point approximation in which a coherent sum over a reduced
number of complex trajectories was carried out. For the case of direct
electrons, a total of two complex trajectories was sufficient to
generate an \textsc{ati} spectrum that converged to the Keldysh
model. On the other hand, convergence of the \textsc{ati} spectrum
including electron rescattering proved to be a more intricate problem.


% how it could be extended to a more sophisticated calculation
% beyond linear polarization
The zero-range potential model was implemented to generate an
approximated strong-field ionization spectrum for the $1b_{1}$ and
$1b_{2}$ molecular orbitals of H$_{2}$O. Each molecular orbital was
treated as an independent spherically symmetric orbital with a binding
energy corresponding to the eigenvalue of the field-free problem with
a one-centre \textsc{scf} effective potential. This approach is by no
means an attempt to provide a realistic picture of the H$_{2}$O
molecule subjected to an intense laser field. However, it could serve
as reference to future works that focus on the convergence of the
ionization spectrum.































































%%% Local Variables:
%%% mode: latex
%%% TeX-master: "thesis"
%%% End:
