\chapter{Conclusions}
\label{ch:conclusions}

This dissertation has developed a study of the ionization of the
H$_{2}$O molecule valence orbitals under an electric dc field in which
\textsc{nhqm} techniques have been implemented. Additionally, a
systematic study of the \textsc{ati} spectrum for a model-He atom was
carried out following an extension of the \textsc{sfa}. In the
following subsections more detailed and summarized conclusions are
offered for these studies.

\section{H$_{2}$O in an external dc field}
\label{ch:h2o_results}

\subsection{$\mathrm{PDE}$ system with a modified $\mathrm{ECS}$}

% describe the approach in a few sentences

% state that for the 1b1 and 1b2 mos, a spherical effective potential
% was used, and for the 3a1 mo a dependen with the polar angle was
% included 

We have carried out a study of the valence orbitals of the H$_{2}$O
molecule, $1b_{1}$, $1b_{2}$, amd $3a_{1}$, in the presence of an
external electric dc field. As a starting point, an orbital-dependent
effective potential was extracted from a self-consistent solution for
the H$_{2}$O \textsc{mo}s expressed as a single-centre Slater-type
orbital~\cite{Moccia_1964}. The $1b_{1}$ and $1b_{2}$ orbitals were
approximated by spherical orbitals and expressed as linear
combinations of $2p_{x}$ and $2p_{y}$ orbitals, respectively. In the
case of the $3a_{1}$ orbital, $s-p$ type Slater orbitals were
considered in addition to retaining the $2p_{z}$ parts of the
\textsc{mo} expansion.

Then a modified \textsc{ecs} technique was
applied to the radial coordinates 



The tunneling ionization and over-barrier
ionization regimes were explored by finding a numerical solution to
the \textsc{pde} system defined by an effective potential obtained
from single-centre Slater-type orbitals. The exterior complex scaling
parameters and a finite-element resolution parameter were optimized to
guarantee a minimum of two to three significant digits for the
solutions.


% 1b1/1b2 results
The resonance parameters that describe the ionization process,
resonance position and width, were explored over a wide range of
electric field strengths. It was demonstrated how an increase of the
field strength beyond a critical point in the over-barrier region
leads to a crossing between the ionization rates of the $1b_{1}$ and
$1b_{2}$ orbitals. Additional observations of the spherical effective
potential for different field strengths were carried out in order to
shed some light on the interpretation of this behaviour.

% 3a1 results

These calculations should serve as motivation for further studies of
molecular orbitals of water using more sophisticated wave
functions. Without diminishing the results presented here, this work
needs to be extended to deal with laser fields for practical
applications. For the hydrogen molecular ion this was done using
Floquet theory~\cite{Tsog_H2mol_ac_2013}, where some parallels were
found between monochromatic \textsc{ac} and the \textsc{dc}
cases. Experimental observations of strong-field ionization of water
vapour are available for short, intense laser
pulses~\cite{exp_h2o_laser_2008,exp_h2o_laser_2014}. One needs to
solve the time-dependent Schr\"{o}dinger equation for realistic
simulations of these~\cite{Farrell_2011,Falge_2010}.


\subsection{Partial-wave approach with a $\mathrm{CAP}$}
% partial-wave + CAP approach




\section{Above threshold ionization for laser-atom interactions}
\label{ch:ati_results}


























































%%% Local Variables:
%%% mode: latex
%%% TeX-master: "thesis"
%%% End:
