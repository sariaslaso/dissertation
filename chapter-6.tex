\chapter{Conclusions}
\label{ch:conclusions}

This dissertation has developed a study of the ionization of the
H$_{2}$O molecule valence orbitals under an electric dc field in which
\textsc{nhqm} techniques have been implemented. Additionally, a
systematic study of the \textsc{ati} spectrum for a model-He atom was
carried out following an extension of the \textsc{sfa}. In the
following subsections more detailed and summarized conclusions are
offered for these studies.

\section{H$_{2}$O in an external dc field}
\label{ch:h2o_results}

\subsection*{$\mathrm{PDE}$ system with a modified $\mathrm{ECS}$}

% describe the approach in a few sentences

% state that for the 1b1 and 1b2 mos, a spherical effective potential
% was used, and for the 3a1 mo a dependen with the polar angle was
% included 

We have carried out a study of the H$_{2}$O valence orbitals,
$1b_{1}$, $1b_{2}$, and $3a_{1}$, in the presence of an external
electric dc field applied along the symmetry axis of the molecule,
$\mathbf{F} = F_{0}\hat{z}$. As a starting point, an orbital-dependent
effective potential was extracted from a self-consistent solution for
the H$_{2}$O \textsc{mo}s expressed as a single-centre Slater-type
orbital~\cite{Moccia_1964}. The $1b_{1}$ and $1b_{2}$ orbitals were
approximated by spherical orbitals and expressed as linear
combinations of $2p_{x}$ and $2p_{y}$ orbitals, respectively. In the
case of the $3a_{1}$ orbital, $s-p$ type Slater orbitals were
considered in addition to retaining the $2p_{z}$ parts of the
\textsc{mo} expansion.

A modified \textsc{ecs} technique was applied to the radial
coordinates in the Schr\"{o}dinger equation with orbital-dependent
effective potential for each \textsc{mo} as part of solving the
complex eigenvalue problem to obtain the Stark resonance
parameters. By means of a phase factor the scaling was turned on
gradually beyond some distance from the origin of coordinates. The
\textsc{ecs} parameters, $r_{\mathrm{s}}$ and $\Delta r$, as well as
the asymptotic scaling angle, $\chi_{\mathrm{s}}$, were optimized in
order to guarantee a minimum of two to three significant digits in the
solutions. The tunneling ionization and over-barrier ionization
regimes were explored by finding a numerical solution to the
\textsc{pde} system resulting from separating the complex-valued wave
function into real and imaginary parts~\cite{sarias_2016,sarias_2017}.


% 1b1/1b2 results
The resonance parameters that describe the ionization process,
resonance position and width, were explored over a wide range of
electric field strengths. It was demonstrated how an increase of the
field strength beyond a critical point in the over-barrier region
leads to a crossing between the ionization rates of the $1b_{1}$ and
$1b_{2}$ orbitals. Additional observations of the spherical effective
potential for different field strengths were carried out in order to
shed some light on the interpretation of this behaviour.

% 3a1 results
For the $3a_{1}$ \textsc{mo}, the orientation of the external field
was extended to negative values, that is the dc field pointing towards
to oxygen atom fixed at the origin. The relationship of the resonance
paratemers (position and width) to the neighboring valence orbitals
$1b_{1}$ and $1b_{2}$ was explored. Interestingly, the $3a_{1}$
orbital was found to ionize more easily than $1b_{1}$ or $1b_{2}$
irrespective of the field direction along $\hat{z}$.

These calculations should serve as motivation for further studies of
molecular orbitals of water using more sophisticated wave
functions. Without diminishing the results presented here, this work
needs to be extended to deal with laser fields for practical
applications. For the hydrogen molecular ion this was done using
Floquet theory~\cite{Tsog_H2mol_ac_2013}, where some parallels were
found between monochromatic \textsc{ac} and the \textsc{dc}
cases. Experimental observations of strong-field ionization of water
vapour are available for short, intense laser
pulses~\cite{exp_h2o_laser_2008,exp_h2o_laser_2014}. One needs to
solve the time-dependent Schr\"{o}dinger equation for realistic
simulations of these~\cite{Farrell_2011,Falge_2010}.


\subsection*{Partial-wave approach with a $\mathrm{CAP}$}
% partial-wave + CAP approach

Ionization of H$_{2}$O valence orbitals exposed to an external dc
field was addressed from an alternative perspective as well. A
three-centre model potential introduced in a study of ion collisions
with water molecules~\cite{illescas_modelV_2011} was represented as a
partial-wave expansion in a basis of spherical
harmonics~\cite{marko_partialwave}.



% results from comparing the field-free problem with other calculations
% overall remarks of the numerical results, comparison with ECS technique



\section{Above threshold ionization for laser-atom interactions}
\label{ch:ati_results}


























































%%% Local Variables:
%%% mode: latex
%%% TeX-master: "thesis"
%%% End:
