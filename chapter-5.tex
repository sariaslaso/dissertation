\chapter{Saddle point approximation}
\label{cha:sp_approx}

%A semi-classical analysis of ATI which describes the formation of the
%ionization spectrum based on the interference of quantum paths is
%included in Sec.~\ref{sec:spa}. The direct ionization regime is
%considered independently in Sec.~\ref{sec:spa_direct}. The analysis
%presented in this chapter closely follows that
%of~\cite{KopoldOptComm2000}.

%\section{\label{sec:spa} Saddle point approximation}

For laser fields of sufficiently high intensity, the ATI spectrum can
be generated by implementing a saddle point evaluation~[ref spa] of
the multidimensional integral for the transition amplitude obtained in
the previous chapter. This semi-classical approximation provides a
deeper physical insight than the expansion in Bessel functions from
the improved Keldysh approximation and establishes a connection
between the process of ATI of an electron with the concept of quantum
paths, which represent space-time trajectories of the tunneling
electrons. This concept has its origins in the alternative formulation
of quantum mechanics introduced by Feynman in terms of path
integrals~\cite{RevModPhysFeynman}, where the probability amplitude of
a quantum mechanical process can be represented as a coherent
superposition of contributions from all possible spatio-temporal paths
that connect the initial and final state of the system.
% re2019 introduction

% Particularly fruitful was the qua- siclassical analysis of this
% theory in terms of the saddle point method, when one looked at the
% propagation of the electronic wave packet in the time interval
% between the transitions into the continuum and back to the ground
% state. It turns out that within this analysis one can iden- tify
% trajectories of the electron in phase space that con- tribute most
% strongly to particular transition amplitudes, just as it is done
% within the path integral formulation of quantum mechanics [11].We
% call those trajectories quasiclassical, since although they follow
% classical Newtonian dynamics, the dynamics must take place in the
% complex domain to account for tunneling

The analysis presented in this chapter establishes the connection
between the quantum mechanical path integral formalism and the
improved Keldysh approximation discussed in Sec.~\ref{kopold_sfa}. The
transition amplitude that describes the ionization of an electron
under an external laser field is evaluated within the two frameworks,
that in which only direct electrons are considered as well as the case
that incorporates rescattering to the parent ion.

% see pra51 lewenstein

% mention that now that the matrix element for ionization has been
% introduced in the previous chapter, we are going to discuss an
% approximation to obtain the ionization spectrum where the integral
% over time is solved based on the stationary points method

\section{\label{sec:q_paths} Quantum path analysis}
% formalism that derives the equations that describe the quantum paths
% and transition probability within the framework of spa

In the length gauge the compact form of the Volkov state can be
expressed as
\begin{eqnarray}
\label{eq:volkov_Lgauge}
\begin{split}
|\psi_{\mathbf{p}}^{(V)}(t)\rangle & = &
|\mathbf{p} - e\mathbf{A}(t)\rangle e^{-i S_{\mathbf{p}}(t)},
\end{split}
\end{eqnarray}
where $|\mathbf{p} - e\mathbf{A}(t)\rangle$ represents a plane-wave
state and
$S_{\mathbf{p}}(t) = 1/2m \int\limits^{t} d\tau [\mathbf{p} -
e\mathbf{A}(\tau)]^{2}$
denotes the action of the system. Consequently, the Volkov
time-evolution operator can be written down in the form of an
expansion in terms of its Volkov states
\begin{eqnarray}
\label{eq:te_volkov}
\begin{split}
U^{(V)}(t,t') & = & \int d^{3}\mathbf{k}
|\psi_{\mathbf{k}}^{(V)}(t) \rangle
\langle \psi_{\mathbf{k}}^{(V)}(t')|.
\end{split}
\end{eqnarray}

Inserting the expansion~(\ref{eq:te_volkov}) into the matrix
element~(\ref{eq:mp_compact}) and given the time dependence of the
ground state wave function,
$|\psi_{0}(t) \rangle = \exp{(iE_{0}t)} | \psi_{0} \rangle$, we may
write
\begin{eqnarray}
\label{eq:me_action}
\begin{split}
M_{\mathbf{p}} = & \int\limits_{-\infty}\limits^{\infty} dt
\int\limits_{-\infty}\limits^{t} \int d^{3}\mathbf{k}
\ \langle \mathbf{p} - e\mathbf{A}(t) | V | \mathbf{k} - e\mathbf{A}(t) \rangle
\ \langle \mathbf{k} - e\mathbf{A}(t') | V | \psi_{0} \rangle \\
&
\times
\exp \left[i\left(-\frac{1}{2m} \int\limits_{t}\limits^{\infty}
d\tau [\mathbf{p} -e\mathbf{A}(\tau)]^{2} -
\frac{1}{2m} \int\limits_{t'}\limits^{t} d\tau [\mathbf{k} -e\mathbf{A}(\tau)]^{2} +
\int\limits_{-\infty}\limits^{t'} d\tau |E_{0}|
\right)
\right] \\
\sim &
\int\limits_{\infty}\limits^{\infty} dt
\int\limits_{-\infty}\limits^{t} dt'
\int d^{3}\mathbf{k} \exp \left[ iS_{\mathbf{p}}(t, t', \mathbf{k}) \right]
m_{\mathbf{p}}(t, t', \mathbf{k}).
\end{split}
\end{eqnarray}
As one may notice, the action in the exponent,
$S_{\mathbf{p}}(t, t', \mathbf{k})$, consists of three parts which
correspond to the action of the entire system after rescattering,
between ionization and rescattering and before ionization.

% point out the contrast with feynman's path integral
% p-57 chapter
It is revealing to point out the contrast of the ionization
amplitude~(\ref{eq:me_action}) obtained with the strong field
approximation with its analogous representation in terms of Feynman's
theory of path integral. The time evolution operator of the entire
system has the path integral representation
\begin{eqnarray}
\label{eq:te_path}
\begin{split}
U(\mathbf{r}t, \mathbf{r}'t') & = &
\int\limits_{(\mathbf{r}',t')\to(\mathbf{r},t)}
\mathcal{D}\left[ \mathbf{r}(\tau) \right] e^{i S(t, t')},
\end{split}
\end{eqnarray}
where
$S(t, t') = \int\limits_{t'}\limits^{t} d\tau
\mathcal{L}[\mathbf{r}(\tau), \tau]$
is the action calculated along a specific path by integrating the
Lagrangian of the entire system along that path, and the integral
measure denoted by $\mathcal{D}\left[ \mathbf{r}(\tau) \right]$
establishes a coherent sum over all possible paths that connect
$(\mathbf{r}t)$ and $(\mathbf{r}'t')$, independently of whether or not
the paths might be followed by the actual system. In contrast, by
implementing the strong field approximation we have approximated the
exact action of the system at the various stages of the process:
before ionization, in between ionization and rescattering, and after
rescattering, as~(\ref{eq:me_action}) indicates, where the ionization
amplitude is computed through a sum over the exponential of the action
over a five-parameter set of paths, parametrized by the ionization
time $t'$, the rescattering time $t$ and the canonical momentum of the
orbit in between $\mathbf{k}$~\cite{KopoldOptComm2000}.

The five-dimensional set of paths over which the transition
amplitude~(\ref{eq:me_action}) is evaluated can be reduced further by
implementing a saddle point approximation of the integral, in which a
handful of relevant paths remains to be considered. In this process,
the transition amplitude~(\ref{eq:me_action}) is approximated by
expanding the phase about its stationary points, saddle points. The
condition
\begin{eqnarray}
\label{eq:S_stationary}
\begin{split}
\frac{\partial S}{\partial q_{i}} & = & 0
\end{split}
\end{eqnarray}
where $q_{i}(i =1, \dots, 5)$ runs over the five variables $t$, $t'$
and $\mathbf{k}$, leads to the saddle-point
equations~\cite{Lewenstein_1995,KopoldOptComm2000}
\begin{eqnarray}
\label{eq:saddle_eqs}
\begin{split}
\left( \mathbf{k} - e\mathbf{A}(t') \right)^{2} = &
-2m|E_{0}| \\
\left( \mathbf{k} - e\mathbf{A}(t)\right)^{2} = &
\left( \mathbf{p} - e\mathbf{A}(t) \right)^{2} \\
(t - t') \mathbf{k} = & \int\limits_{t'}\limits^{t}
d\tau e\mathbf{A}(\tau).
\end{split}
\end{eqnarray}
The solutions
$(t_{S}(\mathrm{Re}\ t_{S} > \mathrm{Re}\ t'_{S}), t'_{S},
\mathbf{k}_{S})$,
are known as the stationary points of the quasicassical action of the
system, and define the quantum orbits over which the time integral
in~(\ref{eq:me_action}) needs to be carried out. From a physical
perspective, Eqs.~(\ref{eq:saddle_eqs}) ensure the energy conservation
at the time of tunneling, elastic scattering of the electron into its
final state when it returns, and that in fact the electron returns to
its parent ion, respectively. Since $|E_{0}| > 0$
in~(\ref{eq:saddle_eqs}), the condition of energy conservation at the
time of ionization cannot be satisfied for any real time $t'$. As a
consequence, the solutions $(t_{S}, t'_{S}, \mathbf{k}_{S})$ of the
saddle-point equations describe complex orbits which restrains a
straightforward visualization of the trajectories.

%p-56 chapter ATI
The matrix element~(\ref{eq:me_action}) can now be expressed in terms
of the saddle point solutions as
\begin{eqnarray}
\label{eq:Mp_final}
\begin{split}
M_{\mathbf{p}} \sim & \sum\limits_{i} \left( \frac{(2\pi i \hbar)^{5}}
{\mathrm{det} (\partial^{2}S / \partial q_{j} \partial q_{k})_{j,k = 1, \dots, 5}}
\right)^{1/2} \times \exp(i S(t_{S_{i}}, t'_{S_{i}}, \mathbf{k}_{S_{i}})),
\end{split}
\end{eqnarray}
where $q_{i}(i = 1,\dots,5)$ runs over the five variables
$t_{S}, t_{S}'$ and $\mathbf{k}_{S}$. The sum~(\ref{eq:Mp_final})
considers a subset of trajectories which determine the shape of the
ionization spectrum through their interferences, constructive or
destrutive.


\section{\label{sec:spa_results} Results}

\subsection{\label{sec:spa_direct} Direct trajectories}
% numerical results

\subsection{\label{sec:spa_resc} Trajectories with rescattering}

% numerical results






%%% Local Variables:
%%% mode: latex
%%% TeX-master: "thesis"
%%% End:
