% strong field ionization
% nhqm in a study of h2o under dc field

% ati of atoms in a strong laser field with sfa formalism
% ..........................................................


% general problem(s), unanswered questions

%The physics of atoms and molecules in strong fields has been subject
%of numerous investigations in the past decades. Strong-field
%ionization phenomena ...

%relevant phenomena can occur, mention ATI

Atoms and molecules exposed to strong fields of magnitude comparable
to their internal binding forces undergo ionization. This process sets
the ground for multiple strong-field ionization phenomena such as
above threshold ionization~(\textsc{ati}).
%, in which an electron
%initially in its ground state can absorb a large number of photons
%that is higher than the minimum necessary for ionization.
This dissertation addresses two separate ionization problems, the dc
Stark ionization of H$_{2}$O valence orbitals, $1b_{1}$, $1b_{2}$, and
$3a_{1}$, within the framework of non-Hermitian quantum mechanics, and
\textsc{ati} for a model-helium atom as part of a review of a previous
quantitative approach based on the strong field
approximation~(\textsc{sfa}).

% what this dissertation investigates
% calculations/models implemented, comparison with previous
% theoretical calculations

Calculations of the dc ionization parameters, dc Stark shift and
exponential decay rates, for the H$_{2}$O valence orbitals are carried
out by solving the Schr\"{o}dinger equation in the complex domain. Two
independent models are implemented in the study of static ionization
of the molecular orbitals. One in which a spherical effective
potential obtained from a self-consistent calculation of H$_{2}$O
orbital energies is combined with an exterior complex scaling approach
to express the problem as a system of partial differential equations
that is solved numerically using a finite-element method. In the
second approach, a model potential for the H$_{2}$O molecule is
expanded in a basis of spherical harmonics and combined with a complex
absorbing potential that results in a complex eigenvalue problem for
the Stark resonances.

The second part of this investigation is concerned with the study of
\textsc{ati} for atoms subjected to a strong laser field. The
convergence of the ionization spectrum for a model-helium atom is
addressed in a systematic study that is carried out following
Keldysh's formalism of \textsc{sfa}. A generalized compact expression
for the ionization amplitude that incorporates electron rescattering
into the analysis is explored as well. Additionally, a model based on
the concept of quantum paths is implemented in the numerical
evaluation of the \textsc{sfa} transition amplitude. In this analysis,
a coherent sum over all allowed quantum trajectories that render the
action stationary is carried out. This calculation allows to generate
an \textsc{ati} spectrum that converges to the numerical Keldysh
amplitude as the number of trajectories increases.






% both direct transitions and rescattering are analyzed separately


























































%%% Local Variables:
%%% mode: latex
%%% TeX-master: "thesis"
%%% End:
