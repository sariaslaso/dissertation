\chapter{H$_{2}$O in an external electric dc field}
\label{cha:dc_h2o}

% H2O in strong dc fields
% effective potential
% outline of the chapter

The stark resonance parameters, which characterize the shift of the
molecular energy levels under an external dc field, are fundamental in
the study of the strong dc field ionization of molecular orbitals. In
the case of the water molecule, the multicentre nature and,
consequently, the additional degrees of freedom, make the strong dc
field ionization of H$_{2}$O an attractive and challenging problem
from the point of view of a theoretical description as well as
experimentally. Complex variable techniques, such as an exterior
complex scaling~\cite{Simon_1979,Scrinzi_2010} and complex-absorbing
potentiasl~\cite{RissMeyer_1993,Krause_2014}, have been implemented in
order to address the problem of molecular static-field ionization and
compute the induced Stark resonances.

The dc Stark problem for the H$_{2}$O valence orbitals is addressed in
this chapter by the implementation of a modified exterior complex
scaling approach that allows to study the ionization of the molecular
orbitals of H$_{2}$O under a strong dc field. The construction of an
effective potential which reflects the indivual properties of the
orbitals is crucial in this
analysis~\cite{sarias_2016,sarias_2017}. In
Sec.~\ref{ch:h2o_structure}, we formulate the problem with emphasis on
the representation of the molecular orbitals. The exterior complex
scaling formalism and its implementation are introduced in
Sec.~\ref{ch:h2o_ecs} as a crucial step in finding a numerical
solution to the problem. The Stark resonance parameters are then
presented in Sec.~\ref{ch:stark_params}.



%Our study involves deriving an effective potential that reproduces the
%symmetry properties of each molecular orbital in order to study the
%Stark resonances for the water molecule under an external static
%electric field


\section{Molecular structure of H$_{2}$O}
\label{ch:h2o_structure}
% Formulation of the Problem

% representation of the orbitals, fig.1 in the papers (also fig.2 of
% the jphysb), with related comments


\section{Partial differential equation approach to the problem}
\label{ch:h2o_pde}

\subsection{Exterior complex scaling}
\label{ch:h2o_ecs}

\section{Stark resonance parameters}
\label{ch:stark_params}
\subsection{$1b_{1}$ and $1b_{2}$ molecular orbitals}
\subsection{$3a_{1}$ molecular orbital}

%%% Local Variables:
%%% mode: latex
%%% TeX-master: "thesis"
%%% End:
