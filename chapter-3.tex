\chapter{H$_{2}$O in an external electric dc field}
\label{cha:dc_h2o}

% H2O in strong dc fields
% effective potential
% outline of the chapter

The stark resonance parameters, which characterize the shift of the
molecular energy levels under an external dc field, are fundamental in
the study of the strong dc field ionization of molecular orbitals. In
the case of the water molecule, the multicentre nature and,
consequently, the additional degrees of freedom, make the strong dc
field ionization of H$_{2}$O an attractive and challenging problem
from the point of view of a theoretical description as well as
experimentally. Complex variable techniques, such as an exterior
complex scaling~\cite{Simon_1979,Scrinzi_2010} and complex-absorbing
potentiasl~\cite{RissMeyer_1993,Krause_2014}, have been implemented in
order to address the problem of molecular static-field ionization and
compute the induced Stark resonances.

The dc Stark problem for the H$_{2}$O valence orbitals is addressed in
this chapter by the implementation of a modified exterior complex
scaling approach that allows to study the ionization of the molecular
orbitals of H$_{2}$O under a strong dc field. The construction of an
effective potential, which reflects the indivual properties of the
orbitals, is crucial in this analysis. In Sec.~\ref{ch:h2o_structure},
we formulate the problem with emphasis on the representation of the
molecular orbitals. The exterior complex scaling formalism and its
implementation are introduced in Sec.~\ref{ch:1b1_1b2} and
Sec.~\ref{ch:3a1} as a crucial step in finding a numerical solution to
a spherically symmetric problem, in the case of the $1b_{1}$ and
$1b_{2}$ orbitals, and the problem of a non-central effective
potential, in the case of the $3a_{1}$ orbital. The Stark resonance
parameters are then presented in Sec.~\ref{ch:stark_params}, in which
the symmetry properties of the orbitals are considered
independently. The analysis presented in this chapter compiles that
of~\cite{sarias_2016,sarias_2017}.


\section{Molecular structure of H$_{2}$O}
\label{ch:h2o_structure}
% Formulation of the Problem

The starting point for this study is the Hartree-Fock (\textsc{hf})
calculation of the H$_{2}$O molecular states using a single-center
Slater orbital
basis~\cite{Moccia_1964,Moccia_JCP_2164,Moccia_JCP_2176}, applied to
collision studies~\cite{Montanari_2013} and compared to experimental
electron spectroscopy~\cite{Hafied_2007}. Other elaborate descriptions
of the molecular structure of H$_{2}$O, formulated within the
independent-electron approximation by the self-consistent field
(\textsc{scf}) or variational Hartree-Fock method with multicentre
Slater orbitals, proved to be satisfactory in describing the molecular
structure of diatomical molecules~\cite{Pitzer_1968}. On the other
hand, the direct application of these multicentre orbitals for
strong-field studies implies significant computational and
methodological challenges.

The general expression for the basis functions, introduced as a set of
single-center
wavefunctions~\cite{Moccia_1964,Moccia_JCP_2164,Moccia_JCP_2176}, is
defined by a Slater-type orbital (\textsc{sto}),
%
\begin{eqnarray}
  \begin{split}
 f_{n, l, m}(\zeta,r,\theta,\phi) & = & \sqrt{\frac{(2\zeta)^{2n+1}}{(2n)!}}
 r^{n-1} \exp(-\zeta r) S_{l, m}(\theta,\phi),
 \end{split}
\label{eq:sto}
\end{eqnarray}
%
where the angular part $S_{l,m}(\theta,\phi)$ represents real
spherical harmonics. The expansion coefficients and nonlinear
coeficients $\{\zeta_{i}\}$, determined by the Roothaan's
self-consistent-field method~\cite{Moccia_1964,Roothaan_1951}, were
used to construct a reduced form of the radial functions that describe
all the molecular orbitals. More specifically, we are interested in
using a reduced \textsc{sto} expansion to construct an effective
pontential that describes the response of the H$_{2}$O orbitals,
$1b_{1}$, $1b_{2}$ and $3a_{1}$, when applying an electric dc field
along the $z-$axis.

Based on their symmetry properties, separate descriptions of the
molecular orbitals were constructed. The dominant components of the
$1b_{1}$ and $1b_{2}$ orbitals, namely the $np_{x}$ and $np_{y}$
parts, were used to derive spherically symmetric effective
orbital-dependent potentials~\cite{sarias_2016}. While a similar
procedure was implemented to the $3a_{1}$ orbital, i.e., retaining the
$np_{z}$ parts of the \textsc{mo}, the strong asymmetry introduced by
the two protons of the H$_{2}$O molecule located in the $y-z$ plane
leading to significant admixtures of $s-$type Slater orbitals in the
Moccia \textsc{sto}s~\cite{Moccia_1964} would have been neglected with
a spherically symmetric potential. Hence, for the dc ionization
problem of the $3a_{1}$ \textsc{mo} \textsc{sto}s of type $2s$ and
$2p_{z}$ were incorporated in the analysis.



% representation of the orbitals, fig.1 in the papers (also fig.2 of
% the jphysb), with related comments

\section{$1b_{1}$ and $1b_{2}$ molecular orbitals}
\label{ch:1b1_1b2}
\subsection{\textsc{pde} approach in spherical polar coordinates}

\section{$3a_{1}$ molecular orbital}
\label{ch:3a1}
\subsection{Interpolation and Latter correction of the
  non-spherical effective potential}


%\section{Partial differential equation approach to the problem}
%\label{ch:h2o_pde}

%\subsection{Exterior complex scaling}
%\label{ch:h2o_ecs}

\section{Stark resonance parameters}
\label{ch:stark_params}
\subsection{$1b_{1}$ and $1b_{2}$ molecular orbitals}
\subsection{$3a_{1}$ molecular orbital}

%%% Local Variables:
%%% mode: latex
%%% TeX-master: "thesis"
%%% End:
