\chapter{Length gauge}
\label{app:L_gauge}

This appendix introduces the length gauge, a commonly used gauge
choice to describe laser-atom interactions. %put this in perspective

The Hamiltonian for an electron in the presence of an external
electromagnetic field described by the potentials $(\mathbf{A},\phi)$
can be expressed as
%
\begin{eqnarray}
  \label{eq:H_lgauge}
  \begin{split}
    H(t) & = & \frac{1}{2m}(\mathbf{p} - e\mathbf{A})^{2}
    + V(\mathbf{r}) + e\phi,
  \end{split}
\end{eqnarray}
%
where $\mathbf{p} = -i\nabla$, $\nabla$ is the gradient operator, and
$V(\mathbf{r})$ is the binding potential of the atom. In the presence
of an electromagnetic field, the Hamiltonian is frequently decomposed
as
%
\begin{eqnarray}
  \label{eq:H_decomposed}
  \begin{split}
    H(t) & = & H_{0} + H_{I}(t),
  \end{split}
\end{eqnarray}
%
where the operator $H_{0} = \mathbf{p}^{2}/2m + V(\mathbf{r})$ is
known as the field-free Hamiltonian, while $H_{I}(t)$ describes the
interaction of an otherwise free electron with the external field, and
its form depends on the gauge employed. Within the long-wavelength
approximation (\textsc{lwa}), or dipole approximation, which neglects
the space dependence of the electric field and the vector potential,
so that $\mathbf{E}(\mathbf{r}, t) \to \mathbf{E}(t)$ and
$\mathbf{A}(\mathbf{r}, t) \to \mathbf{A}(t)$, one finds that the
divergence of the vector potential vanishes $\nabla\cdot\mathbf{A} =
0$, and the interaction operator takes the form
%
\begin{eqnarray}
  \label{eq:H_dipole}
  \begin{split}
    H_{I}(t) & = & e\phi -
    \frac{e}{m}\mathbf{A}\cdot\mathbf{p} + \frac{e^{2}}{2m}\mathbf{A}^{2}
  \end{split}
\end{eqnarray}
%

A gauge transformation is defined as a change of the electromagnetic
field potentials $(\mathbf{A}, \phi)$ by the gauge function $\Lambda =
\Lambda(\mathbf{r},t)$ such that
%
\begin{eqnarray}
  \label{eq:gauge}
  \begin{split}
    \mathbf{A} & \to & \mathbf{A}' = \mathbf{A} + \nabla\Lambda \\
    \phi & \to & \phi' = \phi - \frac{\partial}{\partial t}\Lambda.
  \end{split}
\end{eqnarray}
%
This transformation leaves the physical fields $\mathbf{E}$ and
$\mathbf{B}$ unaffected as well as the time dependent Schr\"{o}dinger
equation $i\frac{\partial}{\partial t}\Psi = H(t)\Psi$. The length
gauge results from the transformation
%
\begin{eqnarray}
  \label{eq:length_gauge}
  \begin{split}
    \Lambda & = & -\mathbf{A}(t) \cdot \mathbf{r},
  \end{split}
\end{eqnarray}
%
which leads to
%$\mathbf{A}' = 0$ and $\phi' = \frac{\partial}{\partial t}
%\mathbf{A}(t) \cdot \mathbf{r} = -\mathbf{E}(t) \cdot \mathbf{r}$
%
\begin{eqnarray}
  \label{eq:A_length}
  \begin{split}
    \mathbf{A}' = & 0 \\
    \phi' = & \frac{\partial}{\partial t}
    \mathbf{A}(t) \cdot \mathbf{r} = -\mathbf{r} \cdot \mathbf{E}(t) 
  \end{split}
\end{eqnarray}
for the vector and scalar potentials of the external laser field,
respectively. In terms of the transformed potentials, the interaction
operator takes the form $H_{I} = -e \mathbf{r} \cdot \mathbf{E}(t)$.
Therefore, in the length gauge the \textsc{tdse} becomes
%
\begin{eqnarray}
  \label{eq:tdse_length}
  \begin{split}
    i \frac{\partial}{\partial t} \Psi(\mathbf{r}, t) & = & \left[H_{0} -
      e\mathbf{r}\cdot\mathbf{E}(t) \right] \Psi(\mathbf{r}, t).
  \end{split}
\end{eqnarray}
%
The solution of the \textsc{tdse}~(\ref{eq:tdse_length}) is the Volkov
wave function $\Psi^{V}(\mathbf{r}, t)$, which represents a plane wave
and describes an electron in a strong laser field.

%%% Local Variables:
%%% mode: latex
%%% TeX-master: "thesis"
%%% End:
