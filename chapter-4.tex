\chapter{Above threshold ionization in laser-atom and laser-molecule interactions}
\label{cha:ati}

% mention:
%
% -ATI models, keldysh approach pionnering work with the strong field
% approximation
% -QM generalization including rescattering
% -study of the ATI spectrum following the one introduced by kopold

The phenomenon of above threshold ionization~\cite{ATI1979} has been
tackled through diverse approaches, attempts to find numerical
solutions to the time-dependent Schr\"{o}dinger equation
(TDSE)~\cite{muller_tdse1999,scrinzi_tdse1999,Joachain2000} lie among
the earliest ones. A variety of efforts that deal with the complexity
of solving this challenging numerical problem have been successful in
the past. In the same way, complementary approaches to the solution of
the TDSE, such as the so-called Volkov-based
methods~\cite{Faisal_1973,Reiss_1980,Kaminski_1997}, have revealed
their strengths within strong-laser field problems in which a
numerical solution would involve a computationally taxing problem. The
strong-field approximation~\cite{KeldyshSFA}, which lies among these
approaches and considers the binding potential as a perturbation, is
the foundation to the formalism discussed in this chapter.

Sec.~\ref{sec:keldysh} presents an overview of the pioneering work by
Keldysh which introduces the strong-field approximation to describe
the laser ionization of atoms. Next, a generalized approach that
introduces rescattering of the electron back to the vicinity of the
binding potential is included in Sec.~\ref{kopold_sfa}. The ionization
regime of a model He atom under a strong laser field is explored in
Sec.~\ref{sec:kopold_qm} for both scenarios: considering only direct
electrons where the ionization spectrum is reproduced by the Keldysh
amplitude, and using a compact expression for the transition amplitude
that encloses the limiting case of direct trajectories while allowing
electrons to rescatter to the parent ion as well. Additionally, this
study is extended to explore the laser ionization of the $1b_{1}$ and
$1b_{2}$ molecular orbitals of H$_{2}$O in Sec.~\ref{sec:mo_sfa}. The
analysis presented in this chapter closely follows that
of~\cite{Kopold_1997sfa}.



%In addiion to exploring the ionization regime of the He atom in
%Sec.~\ref{sec:kopold_qm}, laser ionization of the $1b_{1}$ and
%$1b_{2}$ molecular orbitals of H$_{2}$O is discussed in
%Sec.~\ref{sec:mo_sfa}.


\section{\label{sec:keldysh} Keldysh formalism}
% cjp on keldysh formalism (sec 2 and 3)

% pra56 becker, lewenstein

% eventually obtain keldysh amplitude, after using the two main
% approximations that lead to eq.7 in pra55 kopold


Theoretical models that describe the interaction between intense laser
fields and atoms go back to Keldysh theory of strong field
approximation. The theoretical framework for the strong-field
approximation introduced by Keldysh properly accounts for multiphoton
ionization and tunneling ionization, which establish two limiting
cases of strong-field ionization, and produced good agreement with
experimental data of electron spectra of ATI for relatively low
energies~\cite{Walker_1994exp}. In this chapter we are concerned with
the numerical evaluation of an improved Keldysh
approximation~\cite{Kopold_1997sfa} that accounts for rescattering and
reveals the complex structure of the ionization spectrum.

The transition amplitude from the ground state of an atom with binding
potential $V(\mathbf{r})$, before the arrival of the laser pulse, into
the scattering state $|\psi_{\mathbf{p}}(t)\rangle$ after the pulse
has passed is given by
\begin{eqnarray}
\label{eq:matrix_element}
\begin{split}
M_{\mathbf{p}} & = & \lim_{t\to\infty,t'\to -\infty}
{\langle \psi_{\mathbf{p}}(t) | U(t,t') | \psi_{0}(t') \rangle}.
\end{split}
\end{eqnarray}
Here it is assumed that in the limit of early times, $t'\to -\infty$,
the exact wave function reduces to the unperturbed wave function
$\psi_{0}(t)$ of the initial ground state. The time-evolution operator
$U(t, t')$ is the solution to the initial-value problem
\begin{eqnarray}
\label{eq:time_evolution}
\begin{split}
\left[ i\partial_{t} - H(t)
\right] U(t,t') = 0,& & U(t',t') = 1,
\end{split}
\end{eqnarray}
and propagates the wave function $|\psi(t)\rangle$ from $t'$ to $t$
under the full Hamiltonian
\begin{eqnarray}
\label{eq:H_ati}
\begin{split}
H(t) & = & -\frac{1}{2} \nabla^{2} + H_{I}(t) + V(\mathbf{r})
\end{split}
\end{eqnarray}
which includes the binding potential, $V(\mathbf{r})$, and the
interaction with the external laser field,
$H_{I}(t)=-\mathbf{r}\cdot\mathbf{E}(t)$, under the dipole
approximation in the length gauge~\cite{Kopold_1997sfa}.

% dyson equations for the evolution operator
% leads to eq.3 (pra 55) or eq.14 (cjp)
% follow steps from pra 55 to obtain eq.7 (pra 55)

The time-evolution operator satisfies an integral equation, namely the
Dyson equation~\cite{Kopold_1997sfa,cjp2010_keldysh}, which
conveniently allows to expand the total wave function in terms of the
interaction with the external field $H_{I}(t)$,
\begin{eqnarray}
\label{eq:dyson}
\begin{split}
U(t,t') & = & U_{0}(t,t') -
i\int\limits_{t'}\limits^{t} dt''\ U_{0}(t,t'') H_{I}(t'') U(t'',t') \\
& = & U_{0}(t,t') -
i\int\limits_{t'}\limits^{t} dt''\ U(t,t'') H_{I}(t'') U_{0}(t'',t').
\end{split}
\end{eqnarray}
Here $U_{0}(t,t')$ represents the free time-evolution operator that
propagates the field-free atomic ground state wave function
$\psi_{0}(t')$ forward to the time $t$.

Inserting the Dyson equation~(\ref{eq:dyson}) for the evolution
operator, and recalling the orthogonality of the initial ground state
$|\psi_{0}\rangle$ and scattering state $|\psi_{\mathbf{p}}\rangle$ in
the absense of the laser field, the ionization
amplitude~(\ref{eq:matrix_element}) can be written in the form
\begin{eqnarray}
\label{eq:mp_exact}
\begin{split}
M_{\mathbf{p}} & = & -i \lim\limits_{t\to\infty}
\int\limits_{-\infty}\limits^{t} dt' \langle \psi_{\mathbf{p}}(t)
| U(t,t') H_{I}(t') | \psi_{0}(t') \rangle.
\end{split}
\end{eqnarray}

Expression~(\ref{eq:mp_exact}) is still considered an exact form of
the transition amplitude as no approximations have been implemented.
The time-evolution operator in~(\ref{eq:mp_exact}) propagates the
electron from the initial to the final state which includes the
possibility of major excursions of its orbit away from the parent ion.
The first approximation that leads to Keldysh result consists of
replacing the complete time-evolution operator in~(\ref{eq:mp_exact})
by the Volkov time-evolution operator $U^{(V)}$, which propagates the
wave function of a free electron coupled through the interaction
$H_{I}(t)$ to the external field. In other words, the interaction with
the binding potential is considered a perturbation everywhere except
in the initial and final states and the electron no longer feels the
binding potential during the propagation. Equation~(\ref{eq:mp_exact})
now reads
\begin{eqnarray}
\label{eq:mp_volkov}
\begin{split}
M_{\mathbf{p}} & = & -i \lim\limits_{t\to\infty}
\int\limits_{-\infty}\limits^{t} dt' \langle \psi_{\mathbf{p}}(t) |
U^{(V)}(t,t') H_{I}(t') | \psi_{0}(t')\rangle.
\end{split}
\end{eqnarray}
Equation~(\ref{eq:mp_volkov}) embodies the binding potential only in
the initial state $|\psi_{0}\rangle$ and represents the direct
ionization process in which the electron detached from the atom
escapes without further interaction with the atomic core. A more
practical form of the transition amplitude~(\ref{eq:mp_volkov}) can be
obtained by replacing
$H_{I}(t') = \left[ H_{I}(t') + \mathbf{p}^{2}/2m \right] - \left[
  \mathbf{p}^{2}/2m + V\right] + V,$
where $V$ denotes the atomic binding potential, and making use of the
Schr\"{o}dinger equation satisfied by the Volkov time-evolution
operator
\begin{eqnarray}
\label{eq:volkov_schrodinger}
\begin{split}
-i \frac{\partial}{\partial t} U^{(V)}(t, t') & = &
U^{(V)}(t, t') \left( \frac{\mathbf{p}^2}{2m} + H_{I}(t') \right).
\end{split}
\end{eqnarray}
Integrating Eq.~(\ref{eq:volkov_schrodinger}) by parts with respect to
$t'$ and making use of the orthogonality of the initial state and the
final scattering state, Eq.~(\ref{eq:mp_volkov}) can be rewritten as
\begin{eqnarray}
\label{eq:mp_byparts}
\begin{split}
M_{\mathbf{p}} & = & -i \lim\limits_{t\to\infty}
\int\limits_{-\infty}\limits^{t}
dt' \langle \psi_{\mathbf{p}}(t) |
U^{(V)}(t,t') V | \psi_{0}(t')\rangle.
\end{split}
\end{eqnarray}
In order to solve the limit of $t\to\infty$ within Keldysh framework
an additional approximation is introduced in which the scattering
state $\psi_{\mathbf{p}}$ is replaced by the Volkov wave function
\begin{eqnarray}
\label{eq:volkov}
\begin{split}
\psi_{\mathbf{p}}^{(V)}(\mathbf{r},t) & = & (2\pi)^{-3/2}
\exp\left( -\frac{i}{2m} \int\limits^{t} d\tau
(\mathbf{p} - e\mathbf{A}(\tau))^{2}\right),
\end{split}
\end{eqnarray}
which represents the state of a free electron in a laser field with
time-averaged momentum $\mathbf{p}$. This transformation leads to
obtain an equivalent form of the Keldysh
amplitude~\cite{Kopold_1997sfa,Becker_1997}
\begin{eqnarray}
\label{eq:keldysh_amp}
\begin{split}
M_{\mathbf{p}}^{(0)} & = &
-i \int\limits_{-\infty}\limits^{\infty}
dt\ \langle \psi_{\mathbf{p}}^{(V)}(t) | V | \psi_{0}(t) \rangle,
\end{split}
\end{eqnarray}
and is particularly useful when the binding potential $V$ is treated
as a short-range potential.

Generally, the approximation of replacing the time-evolution
propagator $U(t,t')$ by the Volkov propagator $U^{(V)}(t,t')$ gains in
precision the shorter the range of the binding potential and the
higher the intensity of the laser field. In what follows we will
consider the limiting case of zero-range interactions of the form
%
\begin{eqnarray}
\label{eq:zero-range}
\begin{split}
V(\mathbf{r}) & = & \frac{2\pi}{m\mathbf{\kappa}}
\delta(\mathbf{r}) \frac{\partial}{\partial r} r,
\end{split}
\end{eqnarray}
%
which support a scattering state that approaches a plane wave except
for an $s-$wave
term~\cite{Becker_rescattering1994,Becker_ati2002}. Zero-range
potentials have been widely used in molecular and collision
problems~\cite{zeroV_book1988,frolov_zeroV2013}, as well as
tunneling~\cite{Kleber_zeroV1994} and multiphoton
problems~\cite{Becker_zeroV1990}. Inserting the zero-range
potential~(\ref{eq:zero-range}) into the Keldysh
amplitude~(\ref{eq:keldysh_amp}) yields the expansion
%
\begin{eqnarray}
\label{eq:keldysh_amp_explicit}
\begin{split}
M_{\mathbf{p}}^{(0)} \sim &
\frac{m}{2\pi}\sqrt{2m|E_{0}|}
\sum\limits_{n} \delta\left(\frac{p^{2}}{2m} + U_{p} + |E_{0}| - n\omega\right) \\
& \times \sum\limits_{l=-\infty}\limits^{\infty} J_{2l+n}\left(\frac{2p_{x}}{\omega}
\sqrt{\frac{U_{p}}{m}} \right) J_{l}\left( \frac{U_{p}}{2\omega} \right),
\end{split}
\end{eqnarray}
%
that generates the ionization spectrum of direct electrons
only~~\cite{Kopold_1997sfa}. Here $U_{p}$ represents the ponderomotive
potential of an electron moving in the laser field with momentum
$\mathbf{p}$ parallel to the laser field, $p_{x} = |\mathbf{p}|$,
$|E_{0}|$ stands for the binding energy, and the $J_{n}$ represent
Bessel functions.


\section{\label{kopold_sfa} Generalized ionization amplitude including rescattering}
% pra55 kopold
% derivation of eq.6, generalized of keldysh amplitude

% start mentioning the expansion of the time evolution operator with V
% as a perturbation, which leads to eq.5 in pra55, then mention
% rescattering from then on
In order to include electron rescattering in our study, it is
necessary to allow the electron to interact with the parent ion once
it has been freed from the binding potential. Going back to the exact
expression for the ionization amplitude~(\ref{eq:mp_exact}) and
inserting the Dyson integral equation for the time-evolution operator
\begin{eqnarray}
\label{eq:dysonV}
\begin{split}
U(t,t') & = & U^{(V)}(t,t') -
i\int\limits_{t'}\limits^{t} dt''\ U^{(V)}(t,t'') V U(t'',t'),
\end{split}
\end{eqnarray}
where the binding potential is considered a perturbation, and the
exact time-evolution operator has been replaced by the Volkov
time-evolution operator, one obtains the expression
\begin{eqnarray}
\label{eq:mp_2terms}
\begin{split}
M_{\mathbf{p}} & = & -i \lim\limits_{t\to\infty} \int\limits_{-\infty}\limits^{t}
dt' \langle \psi_{\mathbf{p}}(t) | U^{(V)}(t, t') \{ H_{I}(t') | \psi_{0}(t') \rangle \\
& &
-i \int\limits_{-\infty}^{t'} dt'' V U(t', t'') H_{I}(t'') | \psi_{0}(t'') \rangle \}.
\end{split}
\end{eqnarray}
The first term is the direct amplitude that yields the Keldysh matrix
element discussed in Sec.~\ref{sec:keldysh}. The second term allows
for additional interactions with the atomic potential, and therefore
describes rescattering of the electron. Following the steps
implemented in Sec.~\ref{sec:keldysh} to derive
Eq.~(\ref{eq:keldysh_amp}), the second term of~(\ref{eq:mp_2terms})
results in the compact expression~\cite{Kopold_1997sfa}
\begin{eqnarray}
\label{eq:mp_compact}
\begin{split}
M_{\mathbf{p}} & = & - \int\limits_{-\infty}\limits^{\infty} dt
\int\limits_{-\infty}\limits^{t} dt' \langle \psi_{\mathbf{p}}^{(V)}(t)
| V U^{(V)}(t, t') V | \psi_{0}(t') \rangle,
\end{split}
\end{eqnarray}
where the scattering state was replaced by a plane wave in order to
carry out the limit of $t\to\infty$. This expression now describes
both the direct electrons that depart from the atom without further
interaction with the binding potential as well as the electrons that
are promoted to the continuum at some time $t'$, and propagate in the
laser field until some later time $t$ when they return to within the
range of the binding potential, whereupon they rescatter into their
final Volkov state.

Evaluation of the matrix element~(\ref{eq:mp_compact}) can be very
cumbersome for a finite-range binding potential. However, it
simplifies noticeably in the limit of a zero-range potential of the
form~(\ref{eq:zero-range}) where the spatial integrations become
trivial. Expanding the Volkov wave function and time-evolution
operator in terms of Bessel functions, one of the remaining
quadratures over time can be carried out and yields the energy
conserving $\delta-$function. Therefore, one quadrature is left to be
carried out numerically,
%
\begin{equation}
\label{eq:Mp_quad}
\begin{split}
M_{\mathbf{p}} \sim &
\sum\limits_{n} \delta\left(\frac{p^{2}}{2m} + U_{p} + |E_{0}| - n\omega\right)
\sum\limits_{l=-\infty}\limits^{\infty}
J_{2l+n}\left( \frac{2p_{x}}{\omega} \sqrt{\frac{U_{p}}{m}} \right) \\
& \times \int\limits_{0}\limits^{\infty} d\tau\ \left( \frac{im}{2\pi\tau} \right)^{3/2}
\left( e^{-i[|E_{0}|\tau + l\delta(\tau)]} \right. \\
& \times \exp\left\{-iU_{p}\tau
\left[1 - \left(\frac{\sin\frac{1}{2}\omega\tau}{\frac{1}{2}\omega\tau}\right)^{2}\right]
\right\} \\
& \left. J_{l}\left(y(\tau)\frac{U_{p}}{\omega}\right)
- J_{l}\left(\frac{U_{p}}{2\omega}\right)
\right),
\end{split}
\end{equation}
%
where the real quantities $y(\tau)$ and $\delta(\tau)$ are defined via
%
\begin{eqnarray}
\label{eq:real_quant}
\begin{split}
y(\tau) e^{-i \delta(\tau)} & = & \frac{1}{2} - i \left(
\sin\omega\tau - \frac{4 \sin^{2}\omega\tau/2}{\omega\tau} \right)
e^{-i\omega\tau}.
\end{split}
\end{eqnarray}

%Equations~(\ref{eq:keldysh_amp_explicit}) and~(\ref{eq:Mp_quad}) are
%the formal results that will be studied numerically in this chapter.


\section{\label{sec:results_qm} Results}
\subsection{\label{sec:kopold_qm} Ionization regime. A systematic study}
% systematic study of the convergence of eq.9 and eq.11 to the final
% spectrum
%\subsection{\label{keldysh_qm} Standard Keldysh matrix element}
% convergence of the standard keldysh amplitude, eq.11 show the
% contrast with the expression that includes rescattering and point
% where rescattering is starting to show, i.e., the value of the
% integrals start to differ

This section is concerned with the study of the ionization spectrum
generated by a strong laser field acting upon an atom with a binding
potential that is approximated as a zero-range potential. The external
laser field is assumed to be turned off in the distant past and
future, $t\to\pm\infty$. With this in mind, we carry out the numerical
evaluation of the transition
amplitudes~(\ref{eq:keldysh_amp_explicit}) and~(\ref{eq:Mp_quad}) in
which we concentrate on the case of a linearly polarized field of the
form
%
\begin{eqnarray}
\label{eq:lp_field}
\begin{split}
\mathbf{A} & = & A_{0}\hat{\mathbf{x}} \cos(\omega t).
\end{split}
\end{eqnarray}
%
Both contributions the one from direct electrons described by the
Keldysh amplitude as well as that from rescattering electrons which
interact one more time with the binding potential are considered when
studying the convergence of the ATI matrix element that generates the
ionization spectrum. Our calculation considers a laser field with
$\hbar\omega = 1.58\ \rm{eV}$ at $10^{15}\ \rm{W}/\rm{cm}^{2}$ acting
upon a He atom with $E_{0} = -0.9\ \rm{a.u.}$ as the binding energy.

The numerical evaluation of the remaining quadrature in
Eq.~(\ref{eq:Mp_quad}) in terms of the travel time is not
straightforward as the convergence of the solution indicates to be
sensitive to the working precision requested. Given that the integrand
is independent of the electron energy, associated with $p_{x}$ in
Eq.~(\ref{eq:Mp_quad}), a fixed value of the Bessel function order $l$
would correspond to a single value of the integral. This allows us to
explore the convergence of the individual integrals that form the sum
over Bessel orders before assembling the results to be summed over the
discrete energies given by $n$. In what follows, we will refer to the
time integral as $F(l)$ by rewriting Eq.~(\ref{eq:Mp_quad}) as
%
\begin{eqnarray}
\label{eq:Mp_rew}
\begin{split}
M_{\mathbf{p}} \sim &
%\sum\limits_{n} \delta\left(\frac{p^{2}}{2m} + U_{p} + |E_{0}| - n\omega\right)
%\sum\limits_{l=-\infty}\limits^{\infty}
%J_{2l+n}\left( \frac{2p_{x}}{\omega} \sqrt{\frac{U_{p}}{m}} \right)\ F(l) \\
  \lim\limits_{|l|_{\rm{max}}\to\infty}
\sum\limits_{n} \delta\left(\frac{p^{2}}{2m} + U_{p} + |E_{0}| - n\omega\right)
\sum\limits_{l=-|l|_{\rm{max}}}\limits^{|l|_{\rm{max}}}
J_{2l+n}\left( \frac{2p_{x}}{\omega} \sqrt{\frac{U_{p}}{m}} \right)\ F(l).
\end{split}
\end{eqnarray}
%

In the process of studying the convergence of $F(l)$, we partitioned
the integration interval into subintervals of $2\pi/\omega$ and
explored the progression of the results as a function of how many
intervals are included in the calculation as well as the working
precision requested. A final interval following the $k-$th interval,
$[2\pi/\omega (k-1), 2\pi/\omega k)$, that extends to $+\infty$ is
  included in the calculation. Additionally, in order to bypass the
  singularity at $\tau = 0$ due to the $1/\tau$ factor in $F(l)$, a
  coordinate transform of the form $x \to \sqrt{\tau}$ is implemented
  so that the integrand converges to a finite value as $\tau$
  approaches zero. This special coordinate transform is suitable only
  for small values of $\tau$ given that losing the $1/\tau$ factor
  would slow down the convergence of the integrand to zero at larger
  times.


% include figure with convergence of the value of the integral as the
% number of intervals increases, comment it before the next paragraph

% include figure of Re[\int ...] for l = 10, 80 as a function of the
% working precision for n = 512 intervals (number of sub-intervals
% chosen to generate the spectrum)
% working precision: digits of precision maintained in the calculations
Figure~\ref{fig:WP_convergence} illustrates the evolution of punctual
values of $F(l)$ for a set of $l$ values, $|l| = [10, 40, 80]$, as the
working precision is increased. For $l=10$, a working precision of
about $15$ decimal points seems to not affect the evaluation of the
integral. As $l$ increases, the values of the integral deviate from
the initial evaluation until they converge. This happens relatively
quickly for negative values of $l$ for which the graphic indicates
that approximately $25$ digits of precision would be enough to obtain
the converged result. In contrast, for $l>0$ the digits of precision
needed increased to $50$ for $l = 80$.

\begin{figure}
  \centering
  \includegraphics[width=0.75\textwidth]{figures/ch_ATI_SFA/He/n512PG25MR35l_pm104080logRe}
  \caption{Numerical evaluation of the time integral $F(l)$ contained
    in the transition amplitude~(\ref{eq:Mp_rew}) for $|l| = 10, 40,
    80$, indicated in blue, red and magenta respectively, as a
    function of the working precision requested.}
  \label{fig:WP_convergence}
\end{figure}


% include a figure of how the imaginary part converges to zero as the
% working precision is increased


Given that the transition amplitude that describes the rescattering of
an electron to its binding potential~(\ref{eq:Mp_quad}) is a
generalization of the Keldysh
amplitude~(\ref{eq:keldysh_amp_explicit}) one should expect that the
generalized ATI spectrum contains that of direct electrons at low
electron energies. A comparison between Eqs.~(\ref{eq:Mp_rew})
and~(\ref{eq:keldysh_amp_explicit}) illustrates that, for a given
value of $l$, the function $F(l)$ should be proportional to the Bessel
factor $J_{l}\left( \frac{U_{p}}{2\omega} \right)$. This calculation
was carried out for different values of $l$ in order to corroborate
the validity of the aforementioned generalization.

Figure~\ref{fig:integral_keldysh} exhibits a comparison of the
numerical evaluation of $F(l)$ in~(\ref{eq:Mp_rew}) with the simple
Bessel function in~(\ref{eq:keldysh_amp_explicit}) for several sets of
increasing values of $l_{\rm{max}}$. The coefficients that replace the
integral in the Keldysh amplitude were scaled, divided by a factor of
$5$, so it is possible to see the agreement. For negative values of
$l$, at about $l=-30$, the curves begin to differ as the integrals
oscillate around $10^{-6}\ \rm{(arb.\ units)}$ for a range of negative
$l$ values that extends from $l\approx -30$ to $l\approx -60$,
indicating the presence of rescattering as opposed to the case for the
direct transmission, shown as blue dots, from the Keldysh
amplitude. As one might notice, for sufficiently small negative values
of $l$ ($l < -60$) the values of the integral start dropping below,
indicating that convergence of the ionization spectrum for
rescattering electrons is to be expected. As the Bessel order, $l$,
was increased in the evaluation of the quadrature, the working
precision and precision goal were tuned appropriately so the curves
would remain comparable. This is consistent with
Figure~\ref{fig:WP_convergence}, as the order of Bessel functions
increases, a higher working precision is required in order to find a
numerical solution to the quadrature.

%As you can see the first 10 points (Lmin=-80 to -70) are smaller
%compared to the others, so convergence can be expected.

\begin{figure}
\begin{subfigure}[b]{0.33\linewidth}
  \includegraphics[width=\textwidth]{figures/ch_ATI_SFA/He/l30n512WP20PG15MR35vsKeldysh.pdf}
\end{subfigure}
\begin{subfigure}[b]{0.33\linewidth}
  \includegraphics[width=\textwidth]{figures/ch_ATI_SFA/He/l40n512WP40PG25MR35vsKeldysh.pdf}
\end{subfigure}
\begin{subfigure}[b]{0.33\linewidth}
  \includegraphics[width=\textwidth]{figures/ch_ATI_SFA/He/l50n512WP40PG25MR35vsKeldysh.pdf}
\end{subfigure}
\begin{subfigure}[b]{0.33\linewidth}
  \includegraphics[width=\textwidth]{figures/ch_ATI_SFA/He/l60n512WP40PG25MR35vsKeldysh.pdf}
\end{subfigure}
\begin{subfigure}[b]{0.33\linewidth}
  \includegraphics[width=\textwidth]{figures/ch_ATI_SFA/He/l70n512WP40PG25MR35vsKeldysh.pdf}
\end{subfigure}
\begin{subfigure}[b]{0.33\linewidth}
  \includegraphics[width=\textwidth]{figures/ch_ATI_SFA/He/l80n512WP50PG25MR35vsKeldysh.pdf}
\end{subfigure}
\caption{Numerical evaluation of the time integral $F(l)$ contained in
  the transition amplitude~(\ref{eq:Mp_rew}) (red dots) in contrast
  with its analogous Bessel term in the Keldysh amplitude for direct
  transmission (blue dots) for an atom of He as a function of the
  Bessel function order $l$ for increasing values of $l_{\rm{max}}$,
  $l=[ -l_{\rm{max}} , \dots, l_{\rm{max}}]$.}
  \label{fig:integral_keldysh}
\end{figure}

% include convergence of the spectrum that yields from keldysh
% amplitude for direct electrons

% include convergence of ATI spectrum including rescattering with
% remark that it contains the spectrum of direct electrons for low p_x

The ionization spectrum of He for emission parallel to the electric
field of the laser that contains the contribution of direct electrons,
given by the Keldysh amplitude~(\ref{eq:keldysh_amp_explicit}), is
shown in Figure~\ref{fig:keldysh_convergence}. For a given electron
energy, the sum over the Bessel order was extended up to increasing
values of $l_{\rm{max}}$, ranging from $20$ to $50$, in order to
display the convergence of the spectrum in the limit $l\to\infty$. For
$l_{\rm{max}}$ values as low as $20$ and $30$ the final structure of
the spectrum for very small energies, $<1 \rm{U_{p}}$, begins to be
visible. However, more terms need to be considered in the sum over
Bessel functions in order to obtain the converged spectrum. The yield
consisting only of direct electrons converges relatively fast to its
final shape (dash-dotted line) in which a sequence of narrow
suppressions of the probability amplitude separated by rounded tops
drops as the electron energy increases and eventually vanishes at
about $2.5\rm{U_{p}}$.

\begin{figure}
  \centering
  \includegraphics[width=0.75\textwidth]{figures/ch_ATI_SFA/He/l20304050Keldysh.pdf}
\caption{ATI spectrum of helium by a linearly polarized field
  describing direct electrons. Each curve corresponds to a finite
  value of $l_{\rm{max}}$ in the standard Keldysh amplitude.}
\label{fig:keldysh_convergence}
\end{figure}

The results of the calculations based on~(\ref{eq:Mp_quad}) are shown
in Figure~\ref{fig:mp_convergence}. Each coloured curve represents the
ionization amplitude for an atom of He under a strong laser field for
increasing values of the Bessel function order, $l$. As one might
notice, the ionization spectrum converges for $l=80$ (bottom right
plot) after undergoing some fluctuations for $l$ values between $40$
and $70$. The spectrum for direct electrons (black dots) is included
as a reference. As it can be seen, both the standard Keldysh amplitude
and the fully quantum mechanical result that incorporates rescattering
exhibit very similar electron yields for energies lower than
$2.5\rm{U_{p}}$ where the spectrum is consisting only of direct
electrons. As the electron energy increases, the rescattered electrons
begin to exceed the direct ones and the curves start to differ from
each other. The transition probability, consisting almost exclusively
of rescattered electrons, reaches a plateau consisting of a sequence
of suppressions separated by rounded tops. This behaviour is a direct
consequence of quantum interference, as the released electrons
interfere constructively and destructively in every optical cycle of
the laser field as a function of energy. For large energies of about
$10\rm{U_{p}}$ the plateau shows a cutoff that indicates the end of
the rescattering spectrum. The position of this cutoff as well as the
onset energy of the plateau fluctuate with the orientation of the
emitted electrons with respect to the electric field of the laser as
well as with variations of the intensity of the
field~\cite{Kopold_1997sfa,Paulus_1994plateau,Paulus_1994plateau_classical}.


\begin{figure}
\begin{subfigure}[b]{0.5\linewidth}
  \includegraphics[width=\textwidth]{figures/ch_ATI_SFA/He/l4050n512WP40PG25MR35vsKeldysh.pdf}
\end{subfigure}
\begin{subfigure}[b]{0.5\linewidth}
  \includegraphics[width=\textwidth]{figures/ch_ATI_SFA/He/l5060n512WP40PG25MR35vsKeldysh.pdf}
\end{subfigure}
\begin{subfigure}[b]{0.5\linewidth}
  \includegraphics[width=\textwidth]{figures/ch_ATI_SFA/He/l6070n512WP40PG25MR35vsKeldysh.pdf}
\end{subfigure}
\begin{subfigure}[b]{0.5\linewidth}
  \includegraphics[width=\textwidth]{figures/ch_ATI_SFA/He/l7080n512WP40PG25MR35vsKeldysh.pdf}
\end{subfigure}
\caption{ATI spectrum of a zero-range He model with a binding energy
  of $E_{0} = -0.9\ \rm{a.u.}$ by a linearly polarized field with a
  laser intensity of $10^{15}\ \rm{W/cm^{2}}$ with $\hbar\omega =
  1.58\ \rm{eV}$ in terms of an increasing Bessel order,
  $l_{\rm{max}}$, as a function of the electron energy (in
  colour). The result from the standard Keldysh approximation is shown
  as the black dotted line.}
  \label{fig:mp_convergence}
\end{figure}


\subsection{\label{sec:mo_sfa} Ionization spectrum for the $1b_{1}$ and $1b_{2}$ orbitals
  of H$_{2}$O}
% eq.9 and 11 for 1b1 and 1b2 MOs

% mention studies of the h2o molecule under a laser field, put it in
% terms of wanting to explore the molecular orbital response


% mention that 1b1 and 1b2 MOs were modelled previously as spherical
% orbitals, where only a radial dependence was included in the
% effective potential, under this approximation we extended our study
% to model how this approximate orbitals would ionize in the presence
% of a strong laser field
The study on the H$_{2}$O molecular orbitals presented in
Chapter~\ref{cha:dc_h2o} is extended in this section with the aim of
exploring the ATI spectrum of the $1b_{1}$ and $1b_{2}$ molecular
orbitals previously characterized as spherical orbitals. The
zero-range model calculation carried out in the previous section
combined with the strong field approximation is applied to these
valence orbitals in order to explore their response to an intense
laser field.

%With the aim of exploring the ionization regime of the water molecule
%in terms of the molecular orbital response to an intense laser field,
%the zero-range model calculation discussed in the previous section, in
%which the binding potential of the atom is considered to act on the
%scattered electrons as a delta function, is applied to the H$_{2}$O
%valence orbitals $1b_{1}$ and $1b_{2}$.

% mention more the details of the calculation
Each molecular orbital is treated as an independent atom in which the
eigenvalues $\epsilon_{1b_{1}}$ and $\epsilon_{1b_{2}}$ obtained from
the radial representation of their effective potentials,
$V_{\rm{eff}}(r)$, are considered their binding energies,
respectively. With this in mind, it is possible to generate the
ionization spectrum for direct electrons and that for rescattering
electrons that would correspond to each molecular orbital under a
strong laser field. Inserting the molecular binding energies into
Eqs.~(\ref{eq:keldysh_amp_explicit}) and~(\ref{eq:Mp_quad}) one can
explore the convergence of the ionization spectrum in terms of the
number of Bessel functions included in their respective sums.

% describe the plots comparing direct and rescattering electrons
% describe convergence of the spectrum
Similarly to the case of strong field ionization of a He atom, the
quadrature $F(l)$ in~(\ref{eq:Mp_rew}) remains to be solved in order
to obtain the ionization spectrum for rescattered electrons. The
general expression~(\ref{eq:Mp_quad}), which encloses the limiting
case of ionization of direct electrons, generates an electron yield
which follows that of direct electrons for low energies, i.e.,
electrons energies for which the direct spectrum is not vanished and
the rescattering effects are not taken into account. This section is
aimed to validate the previous statement and explore the convergence
of the ATI spectrum of these two simplified representations of
H$_{2}$O orbitals.

Figures~\ref{fig:1b1_vs_keldysh} and~\ref{fig:1b2_vs_keldysh} show the
values taken by the function $F(l)$ for a set of values of
$l_{\rm{max}}$, $l_{\rm{max}} = 30,\dots,80$, that indicate the
extension of the sum~(\ref{eq:Mp_quad}) in terms of Bessel functions
and the Bessel term in the standard Keldysh
amplitude~(\ref{eq:keldysh_amp_explicit}) in red and blue,
respectively. The numerical values of the integral $F(l)$ were
rescaled for both molecular orbitals, divided by a factor of $5.5$ for
the $1b_{1}$ MO and by a factor of $6.5$ for the $1b_{2}$ MO, in order
to make the comparability between the curves visible. Correspondingly,
the working precision of the calculations was gradually increased for
$|l| > 0$ up to a maximum of $50$ digits of precision for
$l_{\rm{max}} = 80$. As it has been observed for ionization along the
electric field of the laser for a He atom~\cite{Kopold_1997sfa}, the
precise agreement between the emission rate for direct electrons and
the full ionization spectrum including rescattering for energies below
the cutoff of the direct-electron spectrum indicates that a
correlation between the red and blue curves should be expected for a
range of values of $l_{\rm{max}}$ before deviations due to
rescattering become substantial. This behaviour can be observed for
both molecular orbitals for $l < -30$, where the quadrature $F(l)$
reaches a plateau at about $10^{-5}$ that extends up to about $l <
-60$ where signs of convergence of the time integral $F(l)$ become
noticeable as the red curve begins to decline.

\begin{figure}
  \begin{subfigure}[b]{0.33\linewidth}
    \includegraphics[width=\textwidth]{figures/ch_ATI_SFA/1b1/l30n512WP20PG15MR35vsKeldysh.pdf}
  \end{subfigure}
  \begin{subfigure}[b]{0.33\linewidth}
    \includegraphics[width=\textwidth]{figures/ch_ATI_SFA/1b1/l40n512WP20PG15MR35vsKeldysh.pdf}
  \end{subfigure}
  \begin{subfigure}[b]{0.33\linewidth}
    \includegraphics[width=\textwidth]{figures/ch_ATI_SFA/1b1/l50n512WP40PG25MR35vsKeldysh.pdf}
  \end{subfigure}
  \begin{subfigure}[b]{0.33\linewidth}
    \includegraphics[width=\textwidth]{figures/ch_ATI_SFA/1b1/l60n512WP40PG25MR35vsKeldysh.pdf}
  \end{subfigure}
  \begin{subfigure}[b]{0.33\linewidth}
    \includegraphics[width=\textwidth]{figures/ch_ATI_SFA/1b1/l70n512WP40PG25MR35vsKeldysh.pdf}
  \end{subfigure}
  \begin{subfigure}[b]{0.33\linewidth}
    \includegraphics[width=\textwidth]{figures/ch_ATI_SFA/1b1/l80n512WP50PG25MR35vsKeldysh.pdf}
  \end{subfigure}
  \caption{Numerical evaluation of the time integral $F(l)$ in the
    transition amplitude~(\ref{eq:Mp_rew}) (red dots) in contrast with
    its analogous Bessel term in the Keldysh amplitude for direct
    transmission (blue dots) for the $1b_{1}$ MO of H$_{2}$O as a
    function of the Bessel function order $l$ for increasing values of
    $l_{\rm{max}}$, $l=[ -l_{\rm{max}} , \dots, l_{\rm{max}}]$.}
    \label{fig:1b1_vs_keldysh}
\end{figure}

\begin{figure}
  \begin{subfigure}[b]{0.33\linewidth}
    \includegraphics[width=\textwidth]{figures/ch_ATI_SFA/1b2/l30n512WP20PG15MR35vsKeldysh.pdf}
  \end{subfigure}
  \begin{subfigure}[b]{0.33\linewidth}
    \includegraphics[width=\textwidth]{figures/ch_ATI_SFA/1b2/l40n512WP20PG15MR35vsKeldysh.pdf}
  \end{subfigure}
  \begin{subfigure}[b]{0.33\linewidth}
    \includegraphics[width=\textwidth]{figures/ch_ATI_SFA/1b2/l50n512WP40PG25MR35vsKeldysh.pdf}
  \end{subfigure}
  \begin{subfigure}[b]{0.33\linewidth}
    \includegraphics[width=\textwidth]{figures/ch_ATI_SFA/1b2/l60n512WP40PG25MR35vsKeldysh.pdf}
  \end{subfigure}
  \begin{subfigure}[b]{0.33\linewidth}
    \includegraphics[width=\textwidth]{figures/ch_ATI_SFA/1b2/l70n512WP40PG25MR35vsKeldysh.pdf}
  \end{subfigure}
  \begin{subfigure}[b]{0.33\linewidth}
    \includegraphics[width=\textwidth]{figures/ch_ATI_SFA/1b2/l80n512WP50PG25MR35vsKeldysh.pdf}
  \end{subfigure}
  \caption{Numerical evaluation of the time integral $F(l)$ in the
    transition amplitude~(\ref{eq:Mp_rew}) (red dots) in contrast with
    its analogous Bessel term in the Keldysh amplitude for direct
    transmission (blue dots) for the $1b_{2}$ MO of H$_{2}$O as a
    function of the Bessel function order $l$ for increasing values of
    $l_{\rm{max}}$, $l=[ -l_{\rm{max}} , \dots, l_{\rm{max}}]$.}
    \label{fig:1b2_vs_keldysh}
\end{figure}

The ionization spectra corresponding to the $1b_{1}$ and $1b_{2}$
molecular orbitals are shown in Figures~\ref{fig:1b1_spectrum}
and~\ref{fig:1b2_spectrum} as a function of the electron energy. The
evolution of the electron yield is presented in terms of the Bessel
order $l$, $40 \leq l \leq 80$. As it can be noticed, expanding the
sum in Eq.~(\ref{eq:Mp_quad}) up to $l_{\rm{max}} = 80$, purple curve,
leads to convergence of the ATI spectrum for both molecular
orbitals. Consistently with the comparison with the standard Keldysh
amplitude shown in Figures~\ref{fig:1b1_vs_keldysh}
and~\ref{fig:1b2_vs_keldysh}, as $l$ increases a higher working
precision is needed to obtain an accurate representation of the
transmission amplitude. It can be seen that the final shape of the
spectrum for low energies can be obtained for $l$ values as low as
$40$. For those energy values one obtains full agreement between the
transmission due to direct electrons only (black curve) and the
spectrum of rescattered electrons. As the electron energy increases,
the Keldysh amplitudes corresponding to both orbitals $1b_{1}$ and
$1b_{2}$ vanish, giving rise to the onset of the plateau that
describes the spectrum consisting entirely of rescattered electrons.


\begin{figure}
  \centering
  \includegraphics[width=0.75\textwidth]
                  {figures/ch_ATI_SFA/1b1/l40to80n512WP50PG25MR35vsKeldysh.pdf}
  \caption{ATI spectrum for the $1b_{1}$ MO of H$_{2}$O by a linearly
    polarized field with laser intensity of $10^{15}\ \rm{W/cm^{2}}$
    with $\hbar\omega = 1.58\ \rm{eV}$ in terms of an increasing
    Bessel order, $l$, as a function of the electron energy (in
    colour). The result from the standard Keldysh approximation is
    shown as the black dotted line.}
  \label{fig:1b1_spectrum}
\end{figure}

\begin{figure}
  \centering
  \includegraphics[width=0.75\textwidth]
                  {figures/ch_ATI_SFA/1b2/l40to80n512WP50PG25MR35vsKeldysh.pdf}
  \caption{ATI spectrum for the $1b_{2}$ MO of H$_{2}$O by a linearly
    polarized field with laser intensity of $10^{15}\ \rm{W/cm^{2}}$
    with $\hbar\omega = 1.58\ \rm{eV}$ in terms of an increasing
    Bessel order, $l$, as a function of the electron energy (in
    colour). The result from the standard Keldysh approximation is
    shown as the black dotted line.}
  \label{fig:1b2_spectrum}
\end{figure}

%As mentioned above

% more thorough description of the molecular orbital is needed



%%% Local Variables:
%%% mode: latex
%%% TeX-master: "thesis"
%%% End:
