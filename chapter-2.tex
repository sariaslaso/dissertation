\chapter{Electronic Structure of H$_{2}$O}
\label{cha:scf_h2o}

% read
% references -> Rootham-Hartree-Fock -> RoothaanRevModPhys(1951).pdf
% books -> Levine -> ch.15, ch.11, ch.14.3

% mention the Ellison and Shull calculations on H2O that are reference

% initial calculations of the water molecule carried out by means of a
% multicentre HF evaluation of Slater-type orbitals

% approach implemented by Moccia, described in JChemPhys40_2164

This chapter presents a brief compilation of the principles of the
\textsc{hf} formulation implemented in problems of molecular quantum
mechanics. Section~\ref{ch:var_hf} summarizes the Roothaan formulation
of the \textsc{hf} formalism, in which the \textsc{hf} orbitals are
expressed as linear combinations of suitable analytical
functions~\cite{Roothaan_HF}. For molecules, the calculation of
electronic wave functions is more intricate than for atoms since it is
preferable to use basis functions centred about several nuclei, which
implies the difficult task of evaluating multicentre
integrals~\cite{Pitzer_1968,Pitzer_1970}. Sec.~\ref{ch:scf_sto}
describes a different approach that consists of using a set of basis
functions all referred to one common origin and has reported
satisfactory results in calculations of \textsc{scf} molecular
orbitals for AH$_{n}$-type molecules~\cite{Moccia_JCP_2164,
  Moccia_1964}.


\section{Variational Hartree-Fock method}
\label{ch:var_hf}
% HF self-consistent field method to determine molecular orbitals

% define starting point to determine the fock equations

% define AP (antisymmetrized product)

% state fosk's equations

% describe Roothaan's SCF method for a closed-shell ground state
% (linear combination of atomic orbitals)

As a starting point in the \textsc{scf} \textsc{mo} calculations, the
$N-$electron \textsc{hf} wave function for a molecule is written as
the antisymmetrized product (\textsc{ap}) of $N$ one-electron wave
functions~\cite{Roothaan_HF}. A molecular spin-orbital~(\textsc{mso})
which factors into a molecular orbital, $\varphi$, and a spinor,
$\eta$, that depends on the space and spin coordinates of the $\mu$th
electron can be defined as
%
\begin{eqnarray}
  \psi_{k}^{\mu} & = & \varphi_{i(k)}(x^{\mu}, y^{\mu}, z^{\mu})
  \eta_{k}(s^{\mu}),
  \label{eq:mso}
\end{eqnarray}
%
where the superscript $\mu$ stands for the coordinates of the $\mu$th
electron and the subscripts $k$ and $i$ label the different
\textsc{mso}s and \textsc{mo}s, respectively,

The total $N$-electron wave function is then built up as a the
\textsc{ap} of $N$ molecular spin-orbitals of the
form~(\ref{eq:mso})~\cite{Roothaan_HF}
%
\begin{eqnarray}
  \begin{split}
    \Phi & = & \sqrt{N!}\psi_{1}^{[1} \psi_{2}^{\phantom{[]}2} \cdots \psi_{N}^{N]} =
    (N!)^{-1/2}
    \begin{vmatrix}
      \psi_{1}^{1} & \psi_{2}^{1} & \cdots & \psi_{N}^{1} \\
      \psi_{1}^{2} & \psi_{2}^{2} & \cdots & \psi_{N}^{2} \\
      \cdots & \cdots & \cdots & \cdots \\
      \psi_{1}^{N} & \psi_{2}^{N} & \cdots & \psi_{N}^{N}
    \end{vmatrix}
  \end{split},
  \label{eq:AP}
\end{eqnarray}
%
where the superscripts $[1\ 2 \cdots N]$ indicate that one must
consider all the permutations of the sequence $1\ 2 \cdots N$ such
that the Pauli exclusion principle is satisfied, that is, each
\textsc{mo} $\varphi_{i}^{\mu}$ may occur not more than twice
(corresponding to opposite spin projections) in the product wave
function~(\ref{eq:AP}). The right-hand side of Eq.~(\ref{eq:AP}) is a
Slater determinant of \textsc{mso}s that is constructed by taking an
antisymmetric linear combination of products of spin-orbitals.

For a closed-shell structure, in which the \textsc{ap}~(\ref{eq:AP})
is made up of complete electron shells, the \textsc{hf} method seeks
the \textsc{mo}'s that minimize the variational
energy~\cite{Roothaan_HF}
%
\begin{eqnarray}
  \begin{split}
    E & = & 2\sum\limits_{i} h_{i} + \sum\limits_{i}\sum\limits_{j} (2J_{ij} -
    K_{ij}).
  \end{split}
  \label{eq:HF_energy}
\end{eqnarray}
%
The first sum in Eq.~(\ref{eq:HF_energy}) represents the energy of all
the electrons in the field of the nuclei alone, where $h_{i}$ are the
nuclear-field orbital energies~\cite{Roothaan_HF}. The remaining sum
contains the electronic interactions, in which the Coulomb integrals
$J_{ij}$ and exchange integrals $K_{ij}$ are defined by
%
\begin{subequations}
  \begin{equation} \label{eq:Coulomb}
    J_{ij} = \int \frac{\bar\varphi_{i}^{\mu} \bar\varphi_{j}^{\nu}
      \varphi_{i}^{\mu} \varphi_{j}^{\nu}}{r^{\mu\nu}} dv^{\mu\nu}
  \end{equation}
  %
  \begin{equation} \label{eq:exchange}
    K_{ij} = \int \frac{\bar\varphi_{i}^{\mu} \bar\varphi_{j}^{\nu}
      \varphi_{j}^{\mu} \varphi_{i}^{\nu}}{r^{\mu\nu}} dv^{\mu\nu}
  \end{equation}
\end{subequations}
%
where the integration goes over the spatial coordinates of the
$\mu$th and the $\nu$th electron.

The \textsc{hf} \textsc{scf} method looks for those molecular orbitals
$\varphi_{i}$ that minimize the variational
energy~(\ref{eq:HF_energy}). In an iterative process, the molecular
orbitals that form the \textsc{ap}~(\ref{eq:AP}) are corrected by an
infinitesimal amount $\delta\varphi_{i}$, that along with the
requirement that the molecular orbitals continue to form an
orthonormal basis, leads to express the variation of the energy
as~\cite{Roothaan_HF}
%
\begin{eqnarray}
  \begin{split}
    \delta E & = & 2 \sum\limits_{i} \int (\delta\bar\varphi_{i})
    \{ H + \sum\limits_{j} (2J_{j} - K_{j}) \} \varphi_{i} dv +
    2 \sum\limits_{i} \int (\delta\varphi_{i})
    \{ \bar H + \sum\limits_{j} (2\bar J_{j} - \bar K_{j}) \}
    \bar\varphi_{i} dv,
  \end{split}
  \label{eq:delta_Ehf}
\end{eqnarray}
%
where $H = -\frac{1}{2}\nabla^{2} -
\sum\limits_{\alpha}\frac{1}{r^{\alpha}}$ is the Hamiltonian that
represents the field of the nuclei alone. The operators $J_{j}$ and
$K_{j}$ represent the Coulomb operator and exchange operator,
respectively, and are defined as~\cite{Roothaan_HF}
%
\begin{subequations}
  \begin{equation}\label{eq:one_indxJ}
    J_{i}^{\mu} \varphi^{\mu} = \left( \int
    \frac{\bar\varphi_{i}^{\nu} \varphi_{i}^{\nu}}
         {r^{\mu\nu}} dv^{\nu}
         \right) \varphi^{\mu}
  \end{equation}
  %
  \begin{equation}\label{eq:one_indxK}
    K_{i}^{\mu} \varphi^{\mu} = \left( \int
    \frac{\bar\varphi_{i}^{\nu} \varphi_{i}^{\nu}}
         {r^{\mu\nu}} dv^{\nu}
         \right) \varphi^{\mu}.
  \end{equation}
\end{subequations}
%
The Coulomb operator represents the potential energy operator
associated to an electron distributed in space with a density
$|\varphi_{i}|^{2}$. On the other hand, the exchange operator has no
classical analog.

In order for the energy~(\ref{eq:HF_energy}) to reach its absolute
minimum, the condition $\delta E = 0$ must be satisfied for any choice
of \textsc{mo}s that form an orthonormal basis. This condition can be
expressed as~\cite{Roothaan_HF,Levine_QChem}
%
% eq. (14.30) Levine
\begin{eqnarray}
  \begin{split}
    \{H + \sum\limits_{j} ( 2J_{j} - K_{j} ) \} \varphi_{i} & = &
    \sum\limits_{j} \varphi_{j} \epsilon_{ji},
  \end{split}
\label{eq:deltaEzero}
\end{eqnarray}
%
where $\epsilon_{ji}$ are the elements of a Hermitian matrix obtained
in the variational procedure of minimizing the
energy~(\ref{eq:HF_energy})~\cite{Roothaan_HF}. Once an initial guess
for the occupied \textsc{mo}s has been introduced, the orbital
energies $\epsilon_{i}$ can be obtained as the energy eigenvalues of
the Fock operator, $F$, that satisfies
%
\begin{eqnarray}
  \begin{split}
    F & = & H +
    \sum\limits_{j} [ 2 J_{j} - K_{j} ]
  \end{split}
  \label{eq:F_operator}
\end{eqnarray}
%
Consequently, the \textsc{hf} \textsc{scf} problem can be addressed as
the problem of finding the best set of molecular orbitals that
satisfy~\cite{Roothaan_HF,Levine_QChem}
%
\begin{eqnarray}
  \begin{split}
    F \varphi_{i} & = & \sum\limits_{j} \varphi_{j} \epsilon_{ji}.
  \end{split}
  \label{eq:Fock_operator_problem}
\end{eqnarray}
%
Without loss of generality, Eq.~(\ref{eq:Fock_operator_problem}) can
be expressed as the eigenvalue problem
% by means of a unitary transformation (eq.43 roothaan)
%
\begin{eqnarray}
  \begin{split}
    F \varphi_{i} & = & \epsilon_{i} \varphi_{i},
  \end{split}
  \label{eq:Fock_equation}
\end{eqnarray}
%
where $\epsilon_{i}$ are the real elements of a diagonal matrix, which
results from applying a unitary transformation to the Hermitian matrix
$\epsilon_{ji}$. The set of Eqs.~(\ref{eq:Fock_equation}), commonly
known as \textsc{hf} equations, states that the eigenfunctions of the
Hermitian operator $F$ are the set of \textsc{mo}s that give the best
\textsc{ap}, while the eigenvalues $\epsilon_{i}$ represent the
\textsc{hf} orbital energies~\cite{Roothaan_HF}.

% reference other works that used this (Roothaan)formalism
The formalism introduced by Roothaan~\cite{Roothaan_HF}, that allows
the Hartree-Fock-Roothaan orbitals to be expressed as linear
combinations of suitable analytical functions, represents a crucial
development in obtaining accurate numerical results that approximate
the true \textsc{hf} wave functions of the water molecule, as well as
predicting chemical properties in addition to the binding energy of
its ground state
~\cite{scf_lc_1967,EllisonShullh2o_1955,Neumann_gaussian_1968,Pitzer_1970,Pitzer_1972}.
This approach introduces a representation of the molecular orbitals by
means of a linear combination of atomic
orbitals~(\textsc{lcao})~\cite{Roothaan_HF}
%
\begin{eqnarray}
  \begin{split}
    \varphi_{i} & = \sum\limits_{s=1}\limits^{b} c_{si} \chi_{s}.
  \end{split}
  \label{eq:lcao_roothaan}
\end{eqnarray}
%
%p.410 levine
Given that the sum~(\ref{eq:lcao_roothaan}) is an approximation to the
exact \textsc{hf} \textsc{mo}s, the \textsc{ap} built from these
molecular orbitals would be a less good approximation to the exact
\textsc{ap} built from the \textsc{hf} \textsc{mo}s. In the limit of
using an infinitely large basis set, $b\to\infty$, the basis functions
would form a complete set and approximate accurately to the exact
\textsc{hf} \textsc{scf} wave function. The number of basis functions,
$b$, as well as the proper choice of basis functions $\chi_{s}$ are
essential in order to obtain \textsc{mo}s that resemble the
\textsc{hf} \textsc{mo}s with very small
error~\cite{Moccia_JCP_2164,Moccia_JCP_2176,Moccia_1964}.

The problem of obtaining the best set of \textsc{mo}s for a
closed-shell ground state consists in finding the set of coefficients
$c_{si}$ for which the energy of the associated \textsc{ap} reaches
its absolute minimum. The \textsc{lcao} \textsc{scf} procedure begins
with an initial guess for the linear combination of basis
functions~(\ref{eq:lcao_roothaan}). This initial set is used to
compute the Fock operator from equations~(\ref{eq:F_operator})
to~(\ref{eq:one_indxK}). The matrix elements $F_{rs}$
%
\begin{eqnarray}
  \begin{split}
    F_{rs} & \equiv & \langle \chi_{r} | F | \chi_{s} \rangle
  \end{split}
  \label{eq:F_matrix}
\end{eqnarray}
%
are then evaluated in order to determine the nontrivial solutions of
the set of $b$ simultaneous linear equations of the form
%
\begin{eqnarray}
  \begin{split}
    \sum\limits_{s=1}\limits^{b} c_{si} (F_{rs} - \epsilon_{i}S_{rs}) = 0, &
    ~~~~ r = 1,2,\dots,b \\
    S_{rs} \equiv \langle \chi_{r} | \chi_{s} \rangle,~~~~~~~~~~~~
  \end{split}
  \label{eq:set_linear_eqs}
\end{eqnarray}
%
that results from inserting the expansion~(\ref{eq:lcao_roothaan})
into the \textsc{hf} equations~(\ref{eq:Fock_equation}), and
represents a linear algebra generalized eigenvalue problem, in which
the eigenvalues $\epsilon_{i}$ and linear coefficients $c_{si}$ are to
be obtained. The eigenvalues of Eq.~(\ref{eq:set_linear_eqs}), which
are the roots of the secular equation~\cite{Roothaan_HF,Levine_QChem}
%
\begin{eqnarray}
  \begin{split}
    \mathrm{det} (F_{rs} - \epsilon_{i}S_{rs}) = 0
  \end{split}
  \label{eq:secular_eigenvalues}
\end{eqnarray}
%
form an initial set of \textsc{lcao} orbital energies that leads to a
set of coefficients $c_{si}$ and, consequently, \textsc{mo}s.

% CLARIFY WHAT IS BEING ITERATED HERE
In the process of solving the set of
equations~(\ref{eq:Fock_equation}) one looks for the \textsc{mo}s
$\varphi_{i}$ that minimize the variational
energy~(\ref{eq:HF_energy}). Initially, one infers a set of
coefficients $c_{si}$, and, consequently, \textsc{mo}s of the
form~(\ref{eq:lcao_roothaan}), that are used to compute the Fock
operator $F$ and solve the secular
equation~(\ref{eq:secular_eigenvalues}) to obtain an initial set of
orbital energies $\epsilon_{i}$, which in turn is used to obtain an
improved set of coefficients by solving the eigenvalue
problem~(\ref{eq:set_linear_eqs}). This procedure is continued until
the \textsc{mo} coefficients converge according to an established
metric and no further improvement is observed from one evaluation to
the next.

The \textsc{lcao} \textsc{scf} formalism is an approximation that
leads to rather straightforward results for the \textsc{mo}s. This
model leads to accurate approximations of the \textsc{hf} \textsc{scf}
wave function provided the basis set~(\ref{eq:lcao_roothaan}) is large
enough~\cite{EllisonShullh2o_1955, Moccia_JCP_2164, Moccia_JCP_2176,
  Moccia_1964}. Taking things further to achieve a complete
description of the true \textsc{hf} wave function is a much more
complicated mathematical problem.


\section{Self-Consistent Field Slater Orbitals}
\label{ch:scf_sto}

A frequently used set of basis functions in atomic and molecular
\textsc{hf} calculations, in order to represent molecular wave
functions as linear combinations of analytical functions, is the set
of Slater-type orbitals~(\textsc{sto}s) of the form
%
\begin{eqnarray}
  \begin{split}
    f_{n,l,m}(\zeta;r,\theta,\phi) & = &
    \frac{(2\zeta)^{n + \frac{1}{2}}}{[(2n)!]^{\frac{1}{2}}} r^{n-1} e^{-\zeta r}
    S_{l,m}(\theta,\phi),
  \end{split}
  \label{eq:f_STO}
\end{eqnarray}
%
where $n,l,$ and $m$ are integers indicating a basis function, while
the orbital exponent $\zeta$ satisfies $\zeta \in \Re > 0 $. The
angular part $S_{l,m}(\theta,\phi)$ represents real spherical
harmonics.

In order to converge to an exact representation of the \textsc{hf}
orbitals, it would be necessary to include an infinite number of
\textsc{sto}s in an expansion of the
form~(\ref{eq:lcao_roothaan}). However, previous calculations indicate
that it is possible to obtain highly accurate results by using a
finite set of \textsc{sto}s to represent atomic
orbitals~\cite{Clementi_scfIon_1962,Clementi_STOatoms_1974,Bunge_STOtable_1993}
and molecular
orbitals~\cite{Moccia_JCP_2164,Moccia_JCP_2176,Moccia_1964}.

The formalism implemented by Moccia to study the ground state of
XH$_{n}$ molecules~\cite{Moccia_JCP_2164,Moccia_JCP_2176,Moccia_1964}
develops the previously introduced method of using electronic wave
functions expressed by a one-centre expansion with the centre at the X
nucleus~\cite{Parr_JCP_1960,oneCentre_1961}. This approach, labeled as
\textsc{scf} one-centre-expanded molecular
orbitals~\cite{Moccia_JCP_2164}, permits to evaluate the ground state
along with the vibrational spectrum of this type of molecules for a
given geometrical arrange.

Following the Roothaan formalism described in Sec.~\ref{ch:var_hf},
the initial wave function is expressed as an \textsc{ap} of molecular
spin-orbitals in which each molecular orbital is built as a linear
combination of Slater-type orbitals (\textsc{sto}s) of the
form~(\ref{eq:f_STO}). The orbital exponents $\zeta_{i}$ corresponding
to each basis set were obtained by means of an optimization process
closely connected with finding the geometrical equilibrium
configuration of the
molecule~\cite{Moccia_JCP_2164,Moccia_1964}. Returning to the linear
expansion for the $i$-th molecular orbital $\varphi_{i}$ introduced in
Roothaan's formalism, Eq.~(\ref{eq:lcao_roothaan}) can be expressed in
terms of Slater-type orbitals as follows
%
\begin{eqnarray}
  \begin{split}
    \varphi_{i}(\zeta,r,\theta,\phi) & = &
    \sum\limits_{s=1}\limits^{b} c_{si} f_{s_{n,l,m}}(\zeta_{si},r,\theta,\phi),
  \end{split}
  \label{eq:STO_expansion}
\end{eqnarray}
%
where the expansion coefficients $c_{si}$ are evaluated within a
self-consistent procedure in order to minimize the electronic
energy. The sets of $(n,l,m)$ values indicate the size and orbital
symmetries of the basis set implemented for a given molecule.

% mention the expansion of n,l,m values used by Moccia and their
% comments on improving the calculations
The \textsc{scf} calculations for the ground state of H$_{2}$O, in
which the expansion centre was located upon the oxygen
nucleus~\cite{Moccia_1964}, explored different combinations of
symmetries and $n$ values in order to obtain the best possible linear
combination available from a basis set of the
form~(\ref{eq:f_STO}). Table I in~\cite{Moccia_1964} shows the
converged numerical results including the geometrical parameters of
the equilibrium configuration for the ground state of H$_{2}$O. It
emerged from these calculations, in which the basis parameter $n$ was
fixed at values as high as $4$, that very large values of $n$ are not
needed in order to obtain rather accurate results since the converged
wave functions obtained for the \textsc{mo}s could not be dramatically
improved by simply changing the nonlinear parameters of the
\textsc{sto}s, $\zeta_{i}$,~(\ref{eq:f_STO}). Rather, a careful
selection of an initial set for the nonlinear variational parameters
$\zeta_{i}$ for a moderate number of basis functions indicated to be
crucial for improving the radial behaviour of the wave functions for a
given symmetry.

 
% mention that the wave functions could be improved not only by
% changing nonlinear parameters in the STOs but by increasing the size
% of the basis set including higher l values (Moccia JCP-37-910)
As an initial step in the one-centre \textsc{scf} iterative process
implemented by
Moccia~\cite{Moccia_JCP_2164,Moccia_JCP_2176,Moccia_1964}, some
parameters are provided as input data. The basis functions that are
identified by the three integers $n,l,m$ in Eq.~(\ref{eq:f_STO}) and
by the orbital exponents $\zeta$, as well as an estimate of the
expansion coefficients, $c_{si}$, of the molecular orbitals are among
these quantities. The process of variation of the nonlinear parameter
$\zeta$ reveals a strong correlation with the geometrical equilibrium
configuration of the molecule. The numerical results of the one-centre
basis \textsc{scf} method were obtained by modifying the basis
sets~(\ref{eq:lcao_roothaan}) implementing a variational method and
determining the geometrical configuration that minimizes the total
energy for each set of $\zeta$'s. The final results, indicated in
Table I~\cite{Moccia_1964} for the H$_{2}$O molecule, correspond to
the geometrical configuration that provides the lowest total energy
among the different sets considered.


% mention that results cannot be improved drastically by a simple
% change of basis, and what these moccia results were compared to

% discuss briefly the H2O results that Moccia presents and what they
% are compared to


% going beyond a one-centre expansion and implementing a formalism
% closer to the true Hatree-Fock method


%cite pitzer here and their HF calculations
%and obtained near Hartree-Fock wave functions by means of an
%increasing number of multicentre $s$, $p$, and $d$ \textsc{sto}s as
%basis sets~\cite{Pitzer_1968,Pitzer_1970}.

Other studies have addressed the problem of the electronic structure
of H$_{2}$O within the \textsc{hf} \textsc{scf} scheme using
multicentre basis functions, and achieved satisfactory total energies
that approach the converged \textsc{hf} energies rather
well~\cite{natureH2O_1960, gaussianH2O_1965,
  Moccia_oneCenterHF_1967,Reeves_nature_1956}. Many of these
calculations, however, used extensive sets of basis functions or more
complicated forms of wave functions. Gaussian basis sets have also
been used in the construction of \textsc{mo} expansions with the aim
of attacking the problem of multicentre
integrals~\cite{gaussianH2O_1965,Neumann_gaussian_1968}. More advanced
calculations of the H$_{2}$O wave functions in which an increasing
number of multicentre $s$, $p$, and $d$ \textsc{sto}s was used as
basis sets were successful in calculating physical properties, such as
the electric dipole moment and force constants, in good agreement with
experiment~\cite{Pitzer_1968,Pitzer_1970}.


%mention that the scope of this work focuses on the moccia SCF
%calculations with STOs









%%% Local Variables:
%%% mode: latex
%%% TeX-master: "thesis"
%%% End:
