\chapter{Electronic Structure of H$_{2}$O}
\label{cha:scf_h2o}

% read
% references -> Rootham-Hartree-Fock -> RoothaanRevModPhys(1951).pdf
% books -> Levine -> ch.15, ch.11, ch.14.3

% mention the Ellison and Shull calculations on H2O that are reference

% initial calculations of the water molecule carried out by means of a
% multicentre HF evaluation of Slater-type orbitals

% approach implemented by Moccia, described in JChemPhys40_2164

This chapter presents a brief compilation of the principles of the
Hartree-Fock (\textsc{hf}) formulation implemented in problems of
molecular quantum mechanics. Section~\ref{ch:var_hf} summarizes the
principles in the Roothaan formulation of the Hartree-Fock formalism,
in which the Hartree-Fock orbitals are expressed as linear
combinations of suitable analytical functions~\cite{Roothaan_HF}. For
molecules, the calculation of electronic wavefunctions is more
intricate than that for atoms since it is preferable to use basis
functions centred about the several
nuclei~\cite{Pitzer_1968,Pitzer_1970}. This implies the difficult task
of evaluating multicentre integrals~[REFS]. Sec.~\ref{ch:scf_sto}
describes a different approach that consists of using a set of basis
functions all referred to one common origin and has reported
satisfactory results in calculations of self-consistent field
molecular orbitals of AH$_{n}$-type molecules~\cite{Moccia_JCP_2164,
  Moccia_1964}.


\section{Variational Hartree-Fock Method}
\label{ch:var_hf}
% HF self-consistent field method to determine molecular orbitals

% define starting point to determine the fock equations

% define AP (antisymmetrized product)

% state fosk's equations

% describe Roothaan's SCF method for a closed-shell ground state
% (linear combination of atomic orbitals)


As a starting point in the self-consistent field~(\textsc{scf})
molecular orbital~(\textsc{mo}) calculations, the molecular
Hartree-Fock wave function is written as the antisymmetrized product
(\textsc{ap}) of $N$ one-electron wave functions~\cite{Roothaan_HF},
%
\begin{eqnarray}
  \begin{split}
    \Phi & = & \sqrt{N!}\psi_{1}^{[1} \psi_{2}^{\phantom{[]}2} \cdots \psi_{N}^{N]} =
    (N!)^{-1/2}
    \begin{vmatrix}
      \psi_{1}^{1} & \psi_{2}^{1} & \cdots & \psi_{N}^{1} \\
      \psi_{1}^{2} & \psi_{2}^{2} & \cdots & \psi_{N}^{2} \\
      \cdots & \cdots & \cdots & \cdots \\
      \psi_{1}^{N} & \psi_{2}^{N} & \cdots & \psi_{N}^{N}
    \end{vmatrix}
  \end{split},
  \label{eq:AP}
\end{eqnarray}
%
where each spin-orbital $\psi_{k}^{\mu} = \varphi_{i(k)}(x^{\mu},
y^{\mu}, z^{\mu}) \eta_{k}(s^{\mu})$ factors into a molecular orbital
and a spin function; the superscript $\mu$ indicates the spatial and
spin coordinates of the $\mu$th electron, and the subscripts $k$ and
$i$ label the different molecular spin-orbitals (\textsc{mso}'s) and
\textsc{mo}'s, respectively. The superscripts $[1\ 2 \cdots N]$
indicate that one must consider all the permutations of the sequence
$1\ 2 \cdots N$ such that the Pauli exclusion principle is satisfied,
that is, each \textsc{mo} $\varphi_{i}^{\mu}$ may occur not more than
twice (corresponding to opposite spin signs) in the product wave
function.

For a closed-shell structure, in which the \textsc{ap}~(\ref{eq:AP})
is made up of complete electron shells, the Hartree-Fock method looks
for those \textsc{mo}'s that minimize the variational
energy~\cite{Roothaan_HF}
%
\begin{eqnarray}
  \begin{split}
    E & = & 2\sum\limits_{i} H_{i} + \sum\limits_{i}\sum\limits_{j} (2J_{ij} -
    K_{ij}).
  \end{split}
  \label{eq:HF_energy}
\end{eqnarray}
%
The first sum in Eq.~(\ref{eq:HF_energy}) represents the energy of all
the electrons in the field of the nuclei alone, where the Hamiltonian
operator for the $i$th electron is $H_{i} = -\frac{1}{2}
\nabla_{i}^{2} - \sum\limits_{\alpha} \frac{1}{r^{i\alpha}}$. The
remaining sum contains the electronic interactions, in which the
Coulomb integrals $J_{ij}$ and exchange integrals $K_{ij}$ are defined
by
%
\begin{subequations}
  \begin{equation} \label{eq:Coulomb}
    J_{ij} = \int \frac{\bar\varphi_{i}^{\mu} \bar\varphi_{j}^{\nu}
      \varphi_{i}^{\mu} \varphi_{j}^{\nu}}{r^{\mu\nu}} dv^{\mu\nu}
  \end{equation}
  %
  \begin{equation} \label{eq:exchange}
    K_{ij} = \int \frac{\bar\varphi_{i}^{\mu} \bar\varphi_{j}^{\nu}
      \varphi_{j}^{\mu} \varphi_{i}^{\nu}}{r^{\mu\nu}} dv^{\mu\nu}
  \end{equation}
\end{subequations}
%
where the integration goes over the spatial coordinates of the
$\mu$th and the $\nu$th electron

The Hartree-Fock \textsc{scf} method looks for those molecular
orbitals $\varphi_{i}$ that minimize the variational
energy~(\ref{eq:HF_energy}). In an iterative process, the molecular
orbitals that form the \textsc{ap}~(\ref{eq:AP}) are corrected by an
infinitesimal amount $\delta\varphi_{i}$, that along with the
requirement that the molecular orbitals continue to form an
orthonormal basis, leads to express the variation of the energy
as~\cite{Roothaan_HF}
%
\begin{eqnarray}
  \begin{split}
    \delta E & = & 2 \sum\limits_{i} \int (\delta\bar\varphi_{i})
    \{ H + \sum\limits_{j} (2J_{j} - K_{j}) \} \varphi_{i} dv +
    2 \sum\limits_{i} \int (\delta\varphi_{i})
    \{ \bar H + \sum\limits_{j} (2\bar J_{j} - \bar K_{j}) \}
    \bar\varphi_{i} dv
  \end{split}
  \label{eq:delta_Ehf}
\end{eqnarray}
%

In order for the energy~(\ref{eq:HF_energy}) to reach its absolute
minimum, the condition $\delta E = 0$ must be satisfied for any choice
of \textsc{mo}s that conform an orthonormal basis. This condition can
be expressed as~\cite{Roothaan_HF,Levine_QChem}
%
% eq. (14.30) Levine
\begin{eqnarray}
  \begin{split}
    \{H_{i} + \sum\limits_{j} ( 2J_{ij} - K_{ij} ) \} \varphi_{i} & = &
    \sum\limits_{j} \varphi_{j} \epsilon_{ji}
  \end{split}
\label{eq:deltaEzero}
\end{eqnarray}
%
which leads to obtain the orbital energies $\epsilon_{i}$ once an
initial guess for the occupied \textsc{mo}s has been made. One can
define the Hartree-Fock operator $\hat{F}$ by
%
\begin{eqnarray}
  \begin{split}
    \hat{F} & = & \hat{H} +
    \sum\limits_{j} [ 2\hat{J}_{j} - \hat{K}_{j} ]
  \end{split}
  \label{eq:F_operator}
\end{eqnarray}
%
where the Coulomb and exchange operators, defined
in~(\ref{eq:Coulomb}) and~(\ref{eq:exchange}) are expressed by means of
%
\begin{subequations}
  \begin{equation}\label{eq:one_indxJ}
    J_{i}^{\mu} \varphi^{\mu} = \left( \int
    \frac{\bar\varphi_{i}^{\nu} \varphi_{i}^{\nu}}
         {r^{\mu\nu}} dv^{\nu}
         \right) \varphi^{\mu}
  \end{equation}
  %
  \begin{equation}\label{eq:one_indxK}
    K_{i}^{\mu} \varphi^{\mu} = \left( \int
    \frac{\bar\varphi_{i}^{\nu} \varphi_{i}^{\nu}}
         {r^{\mu\nu}} dv^{\nu}
         \right) \varphi^{\mu}
  \end{equation}
\end{subequations}
%
where the integrals are taken over all space. The operator $J_{i}$
represents the potential energy which would arise from an electron
distributed in space with a density $|\varphi_{i}|^{2}$, while the
operator $K_{i}$ has no classical analog. Consequently, the
Hartree-Fock \textsc{scf} problem can be addressed as the problem of
finding the best set of molecular orbitals that
satisfy~\cite{Roothaan_HF,Levine_QChem}
%
\begin{eqnarray}
  \begin{split}
    F \varphi_{i} & = & \sum\limits_{j} \varphi_{j} \epsilon_{ji}.
  \end{split}
  \label{eq:Fock_operator_problem}
\end{eqnarray}
%

% reference other works that used this (Roothaan)formalism
The formalism introduced by Roothaan~\cite{Roothaan_HF}, in which the
Hartree-Fock orbitals are expressed as linear combinations of suitable
analytical functions, represents a crucial development in obtaining
accurate numerical results~[CITE REFS TO THIS] that approximate to the
best \textsc{mo}s of a molecule. In this approach, the molecular
orbitals are expressed by a linear combination of atomic
orbitals~(\textsc{lcao})~\cite{Roothaan_HF}
%
\begin{eqnarray}
  \begin{split}
    \varphi_{i} & = \sum\limits_{s=1}\limits^{b} c_{si} \chi_{s}.
  \end{split}
  \label{eq:lcao_roothaan}
\end{eqnarray}
%
Given that the sum~(\ref{eq:lcao_roothaan}) is an approximation to the
exact Hartree-Fock \textsc{mo}s, the \textsc{ap} built from these
molecular orbitals would be a less good approximation to the exact
\textsc{ap} built from the Hartree-Fock \textsc{mo}s. The number of
basis functions, $b$, as well as the proper choice of basis functions
$\chi_{s}$ are essential in order to obtain \textsc{mo}s that resemble
the Hartree-Fock \textsc{mo}s with negligible
error~\cite{Moccia_JCP_2164,Moccia_JCP_2176,Moccia_1964}.
%p.410 levine

The problem of obtaining the best set of \textsc{mo}s for a
closed-shell ground state consists in finding the set of coefficients
$c_{si}$ for which the energy of the asociated \textsc{ap} reaches its
absolute minimum. The \textsc{lcao} self-consistent field procedure
begins with an initial guess for the linear combination of basis
functions~(\ref{eq:lcao_roothaan}). This initial set is used to
compute the Fock operator from equations~(\ref{eq:F_operator})
to~(\ref{eq:one_indxK}). The matrix elements $F_{rs}$
%
\begin{eqnarray}
  \begin{split}
    F_{rs} & \equiv & \langle \chi_{r} | \hat{F} | \chi_{s} \rangle
  \end{split}
  \label{eq:F_matrix}
\end{eqnarray}
%
are then evaluated in order to determine the nontrivial solutions of
the set of $b$ simultaneous linear equations of the form
%
\begin{eqnarray}
  \begin{split}
    \sum\limits_{s=1}\limits^{b} c_{si} (F_{rs} - \epsilon_{i}S_{rs}) = 0 &
    ~~~~ r = 1,2,\dots,b
  \end{split}
  \label{eq:set_linear_eqs}
\end{eqnarray}
%
that results from inserting the expansion~(\ref{eq:lcao_roothaan})
into the Hartree-Fock equations~(\ref{eq:Fock_operator_problem}). The
solutions $\epsilon_{i}$, which are the roots of the secular
equation~\cite{Roothaan_HF,Levine_QChem}
%
\begin{eqnarray}
  \begin{split}
    \mathrm{det} (F_{rs} - \epsilon_{i}S_{rs}) = 0, &
    ~~~~ S_{rs} \equiv \langle \chi_{r} | \chi_{s} \rangle,
  \end{split}
  \label{eq:secular_eigenvalues}
\end{eqnarray}
%
conform an initial set of \textsc{lcao} orbital energies that leads to
an initial set of coefficients $c_{si}$ and, consequently,
\textsc{mo}s.

The iterative approach of solving Eq.~(\ref{eq:Fock_operator_problem})
is one of trial and error, in which one continues to determine
improved sets of coefficients and orbitals energies, and compares the
resulting set of improved $c_{si}$ with the existing ones. This
process is continued until the \textsc{mo} coefficients converge
according to an established metric and no further improvement is
observed from one iteration to the next.

The \textsc{lcao} self-consistent field formalism is an approximation
that leads to rather straightforward results for the
\textsc{mo}s. This model leads to accurate approximations of the
Hartree-Fock \textsc{scf} wave function provided the basis
set~(\ref{eq:lcao_roothaan}) is large
enough~\cite{EllisonShullh2o_1955, Moccia_JCP_2164, Moccia_JCP_2176,
  Moccia_1964}. Taking things further to achieve a complete
description of the true Hartree-Fock wave function is a much more
complicated mathematical problem.


% DESCRIBE THE STEPS THAT FORM THE HF SCF METHOD

% Initially, one guesses an antisymmetric linear
% combination~(\ref{eq:AP}) followed by a \textsc{scf} iterative
% process until no further improvement is obtained in the wave
% functions.


\section{Self-Consistent Field Slater Orbitals}
\label{ch:scf_sto}

% Slater orbitals in Moccia's calculations following Roothaan procedure
% describe the procedure in Moccia JChemPhys40_2164

The formalism implemented by Moccia to study the ground state of
XH$_{n}$ molecules~\cite{Moccia_JCP_2164,Moccia_JCP_2176,Moccia_1964}
develops the previously introduced method of using electronic wave
functions expressed by a one-centre expansion with the centre at the X
nucleus~\cite{Parr_JCP_1960,oneCentre_1961}. This approach, labeled as
self-consistent field one-centre-expanded molecular
orbitals~\cite{Moccia_JCP_2164}, allows to evaluate the ground state
along with the vibrational spectrum of this type of molecules for a
given geometrical arrange.

Following the Roothaan formalism described in Sec.~\ref{ch:var_hf},
the initial wave function is expressed as an \textsc{ap} of molecular
spin orbitals in which each molecular orbital is built as a linear
combination of Slater-type orbitals (\textsc{sto}s) using basis
functions of the form
%
\begin{eqnarray}
  \begin{split}
    f_{n,l,m}(\zeta;r,\theta,\phi) & = &
    \frac{(2\zeta)^{n + \frac{1}{2}}}{[(2n)!]^{\frac{1}{2}}} r^{n-1} e^{-\zeta r}
    S_{l,m}(\theta,\phi).
  \end{split}
  \label{eq:f_STO}
\end{eqnarray}
%
% indicate what each parameter represents and what are the quantities
% one optimizes in the SCF process of minimizing the total energy
% describe the process or some relevant steps that Moccia mentions
The angular part $S_{l,m}(\theta,\phi)$ in~(\ref{eq:f_STO}) represents
real spherical harmonics and the orbital exponents $\zeta$'s involved
in each basis set were obtained by an optimization process closely
connected with finding the geometrical equilibium configuration of the
molecule~\cite{Moccia_JCP_2164,Moccia_1964}. The linear expansion for
the $i$-th molecular orbital $\varphi_{i}$ introduced in Roothaan's
formalism, Eq.~(\ref{eq:lcao_roothaan}), can be expressed in terms of
Slater-type orbitals as follows
%
\begin{eqnarray}
  \begin{split}
    \varphi_{i}(\zeta,r,\theta,\phi) & = &
    \sum\limits_{s=1}\limits^{b} c_{si} f_{s_{n,l,m}}(\zeta,r,\theta,\phi),
  \end{split}
  \label{eq:STO_expansion}
\end{eqnarray}
%
where the expansion coefficients $c_{si}$ are evaluated within the
iterative process in order to minimize the electronic energy.  The
sets of $(n,l,m)$ values indicate the size of the basis set
implemented for a given molecule.
% mention the expansion of n,l,m values used by Moccia and their
% comments on improving the calculations
The \textsc{scf} calculations for the ground state of H$_{2}$O, in
which the expansion centre was located upon the oxygen
nucleus~\cite{Moccia_1964}, tested different combinations of
symmetries and $n$ values in order to obtain the best possible linear
combination available from a basis set of the
form~(\ref{eq:f_STO}). Table I in~\cite{Moccia_1964} shows the
converged numerical results including the geometrical parameters of
the equilibrium configuration for the ground state of H$_{2}$O. It
emerged from these calculations, in which the highest $n$ quantum
number was set to $4$, that high values of $n$ are not crucial in
order to obtain rather accurate results since the converged wave
functions obtained for the \textsc{mo}s could not be dramatically
improved by simply changing the nonlinear parameters of the
\textsc{sto}s~(\ref{eq:f_STO}).

 
% mention that the wave functions could be improved not only by
% changing nonlinear parameters in the STOs but by increasing the size
% of the basis set including higher l values (Moccia JCP-37-910)
As an initial step in the iterative process, some parameters are
provided as input data. The basis functions that are identified by the
three integers $n,l,m$ in Eq.~(\ref{eq:f_STO}) and by the orbital
exponents $\zeta$, as well as an estimate of the expansion
coefficients, $c_{si}$, of the molecular orbitals are among these
quantities. The process of optimization of the nonlinear parameter
$\zeta$ reveals a strong correlation with the geometrical equilibrium
configuration of the molecule. The numerical results of the one-centre
basis \textsc{scf} method
in~\cite{Moccia_JCP_2164,Moccia_JCP_2176,Moccia_1964} were obtained by
modifying the basis sets~(\ref{eq:lcao_roothaan}) by trial and
determining the geometrical configuration that minimizes the total
energy for each set of $\zeta$'s. The final result being the lowest of
these minima.

% mention that results cannot be improved drastically by a simple
% change of basis, and what these moccia results were compared to

% discuss briefly the H2O results that Moccia presents and what they
% are compared to





% going beyond a one-centre expansion and implementing a formalism
% closer to the true Hatree-Fock method
Other more sofisticated calculations~\cite{Pitzer_1968,Pitzer_1970}
addressed the problem of the electronic structure of the water
molecule and obtained approximated Hartree-Fock wave functions by
means of an increasing number of multicentre $s$, $p$, and $d$
\textsc{sto}s as basis sets.




%%% Local Variables:
%%% mode: latex
%%% TeX-master: "thesis"
%%% End:
