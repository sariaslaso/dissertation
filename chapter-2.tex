
\chapter{Electronic Structure of H$_{2}$O}
\label{cha:scf_h2o}

% read
% references -> Rootham-Hartree-Fock -> RoothaanRevModPhys(1951).pdf
% books -> Levine -> ch.15

% initial calculations of the water molecule carried out by means of a
% multicentre HF evaluation of Slater-type orbitals

% approach implemented by Moccia, described in JChemPhys40_2164

This chapter presents a brief compilation of the principles of the
Hartree-Fock (\textsc{hf}) formulation implemented in problems of
molecular quantum mechanics. Section~\ref{ch:var_hf} summarizes the
principles in the Roothaan formulation of the Hartree-Fock formalism,
in which the Hartree-Fock orbitals are expressed as linear
combinations of suitable analytical functions~\cite{Roothaan_HF}. For
molecules, the calculation of electronic wavefunctions is more
intricate than that for atoms since it is preferable to use basis
functions centred about the several
nuclei~\cite{Pitzer_1968,Pitzer_1970}. This implies the difficult task
of evaluating multicentre integrals. Sec.~\ref{ch:scf_sto} describes a
different approach that consists of using a set of basis functions all
referred to one common origin and has reported satisfactory results in
calculations of self-consistent field molecular orbitals of
AH$_{n}$-type molecules~\cite{Moccia_JCP_2164, Moccia_1964}.




\section{Variational Hartree-Fock Method}
\label{ch:var_hf}
% HF self-consistent field method to determine molecular orbitals

\section{Self-Consistent Field Slater Orbitals}
\label{ch:scf_sto}

% Slater orbitals in Moccia's calculations



%%% Local Variables:
%%% mode: latex
%%% TeX-master: "thesis"
%%% End:
