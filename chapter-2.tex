
\chapter{Electronic Structure of H$_{2}$O}
\label{cha:scf_h2o}

% read
% references -> Rootham-Hartree-Fock -> RoothaanRevModPhys(1951).pdf
% books -> Levine -> ch.15, ch.11

% mention the Ellison and Shull calculations on H2O that are reference

% initial calculations of the water molecule carried out by means of a
% multicentre HF evaluation of Slater-type orbitals

% approach implemented by Moccia, described in JChemPhys40_2164

This chapter presents a brief compilation of the principles of the
Hartree-Fock (\textsc{hf}) formulation implemented in problems of
molecular quantum mechanics. Section~\ref{ch:var_hf} summarizes the
principles in the Roothaan formulation of the Hartree-Fock formalism,
in which the Hartree-Fock orbitals are expressed as linear
combinations of suitable analytical functions~\cite{Roothaan_HF}. For
molecules, the calculation of electronic wavefunctions is more
intricate than that for atoms since it is preferable to use basis
functions centred about the several
nuclei~\cite{Pitzer_1968,Pitzer_1970}. This implies the difficult task
of evaluating multicentre integrals. Sec.~\ref{ch:scf_sto} describes a
different approach that consists of using a set of basis functions all
referred to one common origin and has reported satisfactory results in
calculations of self-consistent field molecular orbitals of
AH$_{n}$-type molecules~\cite{Moccia_JCP_2164, Moccia_1964}.


\section{Variational Hartree-Fock Method}
\label{ch:var_hf}
% HF self-consistent field method to determine molecular orbitals

% define starting point to determine the fock equations

% define AP (antisymmetrized product)

% state fosk's equations

% describe Roothaan's SCF method for a closed-shell ground state
% (linear combination of atomic orbitals)


The starting point in the Hartree-Fock self-consistent field
(\textsc{scf}) method is the antisymmetrized product (\textsc{ap}) of
$N$ one-electron wave functions~\cite{Roothaan_HF},
%
\begin{eqnarray}
  \begin{split}
    \Phi & = & \sqrt{N!}\psi_{1}^{[1} \psi_{2}^{2} \cdots \psi_{N}^{N} =
    (N!)^{-1/2}
    \begin{vmatrix}
      \psi_{1}^{1} & \psi_{2}^{1} & \cdots & \psi_{N}^{1} \\
      \psi_{1}^{2} & \psi_{2}^{2} & \cdots & \psi_{N}^{2} \\
      \cdots & \cdots & \cdots & \cdots \\
      \psi_{1}^{N} & \psi_{2}^{N} & \cdots & \psi_{N}^{N}
    \end{vmatrix}
  \end{split},
  \label{eq:AP}
\end{eqnarray}
%
where the spinorbital $\psi_{k}^{\mu} = \varphi_{i(k)}(x^{\mu},
y^{\mu}, z^{\mu}) \eta_{k}(s^{\mu})$ factors into a molecular orbital
(\textsc{mo}) and a spin function; the superscript $\mu$ indicates the
spatial and spin coordinates of the $\mu$th electron, and the
subscripts $k$ and $i$ label the different molecular spinorbitals
(\textsc{mso}'s) and \textsc{mo}'s, respectively. The superscripts
$[1\ 2 \cdots N]$ indicate that one must consider all the permutations
of the sequence $1\ 2 \cdots N$ such that the Pauli exclusion
principle is satisfied, that is, each \textsc{mo} $\varphi_{i}$ may
occur not more than twice (corresponding to opposite spin signs) in
the product wave function.

For a closed-shell structure, in which the \textsc{ap}~(\ref{eq:AP})
is made up of complete electron shells, the Hartree-Fock method looks
for those \textsc{mo}'s that minimize the variational energy
%
\begin{eqnarray}
  \begin{split}
    E & = & 2\sum\limits_{i} H_{i} + \sum\limits_{ij} (2J_{ij} -
    K_{ij}).
  \end{split}
  \label{eq:HF_energy}
\end{eqnarray}
%
The orbital energies $H_{i}$, the Coulomb integrals $J_{ij}$ and
exchange integrals $K_{ij}$ are defined by~\cite{Roothaan_HF}
%
\begin{eqnarray}
  \begin{split}
    H_{i} & = & \int \bar{\varphi}_{i} H \varphi_{i} dv
  \end{split}
  \label{eq:orbital_Hi}
\end{eqnarray}
%
\begin{eqnarray}
  \begin{split}
    J_{ij} & = & \int \frac{\bar\varphi_{i}^{\mu} \bar\varphi_{j}^{\nu}
      \varphi_{i}^{\mu} \varphi_{j}^{\nu}}{r^{\mu\nu}} dv^{\mu\nu}
  \end{split}
  \label{eq:Coulomb_integral}
\end{eqnarray}
%
\begin{eqnarray}
  \begin{split}
    K_{ij} & = & \int \frac{\bar\varphi_{i}^{\mu} \bar\varphi_{j}^{\nu}
      \varphi_{j}^{\mu} \varphi_{i}^{\nu}}{r^{\mu\nu}} dv^{\mu\nu}
  \end{split}
  \label{eq:exchange_integral}
\end{eqnarray}


Initially, one guesses an antisymmetric linear
combination~(\ref{eq:AP}) followed by a \textsc{scf} iterative process
until no further improvement is obtained in the wave functions. 





\section{Self-Consistent Field Slater Orbitals}
\label{ch:scf_sto}

% Slater orbitals in Moccia's calculations following Roothaan procedure



%%% Local Variables:
%%% mode: latex
%%% TeX-master: "thesis"
%%% End:
